\documentclass[../hydrozoa.tex]{subfiles}
\graphicspath{{\subfix{../images/}}}
\begin{document}

\chapter{Architecture}%
\label{h:architecture}%

\section{Codebase}%
\label{h:architecture-codebase}%

Hydrozoa uses a single mono-repository called \code{cardano-hydrozoa/hydrozoa} \citep{FlerovskyRodionovHydrozoaRepositoryGithub2025}.
Tentatively, we will organize the codebase into the following packages:
\begin{description}
  \item[Spec.] The source code for this specification.
  \item[Multisig.] Onchain scripts, transaction building code, and blockchain queries for the L1 multisig regime (\cref{h:l1-multisig-regime}).
  \item[Plutus.] Onchain scripts, transaction building code, and blockchain queries for the L1 rule-based regime (\cref{h:l1-rule-based-regime}).
  \item[Ledger.] L2 ledger state and transitions (\cref{h:l2-ledger}).
  \item[Blocks.] L2 block creation, validation, and effects (\cref{h:l2-blocks}).
  \item[Head state.] L2 head state, storage, and queries. (\cref{h:l2-head-state}).
  \item[Consensus.] L2 consensus protocol messages, handlers, API (\cref{h:l2-consensus-protocol}).
  \item[Node.] Hydrozoa node software and interface (\cref{h:architecture-processes-and-components}).
  \item[Infra.] Development environment, CI/CD, deployment scripts, and configurations.
\end{description}

\section{Onchain script interdependencies}%
\label{h:ref}%

Hydrozoa's L1 protocol has the following onchain script interdependencies:
\begin{description}
  \item[Native script] A native script parametrized on the list of peers (\cref{h:l1-multisig-regime}).
  \item[Native minting policy.] A minting policy controlled by the native script (\cref{h:l1-multisig-utxo-state}).
  \item[Treasury script.] A Plutus-based spending validator parametrized on the native script (\cref{h:l1-rule-based-treasury}).
  \item[Tallying script.] A Plutus-based staking validator parametrized on the native script (\cref{h:l1-rule-based-dispute-resolution}).
  \item[Voting script.] A Plutus-based spending validator parametrized on the native script and the tallying script (\cref{h:l1-rule-based-dispute-resolution}).
\end{description}

\section{Processes and components}%
\label{h:architecture-processes-and-components}%

In order to operate a Hydrozoa head, each peer will need to run the following processes/software:
\begin{description}
  \item[Hydrozoa UI.] A web-based or terminal-based interface for a user to initialize, fund, and interact with a Hydrozoa head.
  \item[Hydrozoa node.] An executable that manages the peer's L2 state for the head and facilitates the peer's participation in the L2 consensus protocol.
    It provides a backend API for the Hydrozoa UI's interactions with the head.
  \item[User wallet.] A wallet that holds the user's cryptographic keys for manual transaction signing.
    User wallet signatures are needed in the head's initialization and deposit transactions, which bring funds into the head.
  \item[Cardano node (L1).] A Cardano node synchronized to the chain tip of the target L1 network (e.g., Cardano mainnet).
    It is the underlying source of events about the head's L1 state, and the Hydrozoa node submits all L1 transactions to it.
  \item[Blockchain indexer (L1).] A blockchain indexer (e.g., Blockfrost, Ogmios/Kupo) that indexes the events from the Cardano node.
    It responds to the Hydrozoa node's queries about the head's L1 state.
\end{description}

\end{document}
