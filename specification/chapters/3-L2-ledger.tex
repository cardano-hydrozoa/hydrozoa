\documentclass[../hydrozoa.tex]{subfiles}
\graphicspath{{\subfix{../images/}}}
\begin{document}

\chapter{L2 ledger}%
\label{h:l2-ledger}%

The L2 ledger state has this type:
\begin{equation*}
  \T{LedgerState^{L2}} \coloneq \left\{
  \begin{array}{lll}
    \T{utxosActive} &::& \T{UtxoSet^{L2}}
  \end{array}\right\}
\end{equation*}
These fields are interpreted as follows:
\begin{description}
  \item[utxos active.] The set of utxos (\cref{h:l2-utxo-set}) that can be spent in transactions or withdrawn.
    These utxos are created by L2 transactions and genesis events.
\end{description}

\section{Ledger rules}%
\label{h:l2-ledger-rules}%

Hydrozoa's L2 ledger rules are the same as Cardano's ledger rules,%
\footnote{This specification is pinned to this version of the Cardano ledger rules: \citep{IntersectMBOCardanoLedgerV117402025}.}
but with the following additions:
\begin{description}
    \item[No staking or governance actions.]
      Hydrozoa's consensus protocol is not based on Ouroboros proof-of-stake, and its governance protocol is not based on Cardano's hard-fork-combinator update mechanism.
      Furthermore, Hydrozoa's L1 scripts cannot authorize arbitrary staking or governance actions on behalf of users.
      For this reason, staking and governance actions cannot be used in L2 transactions.%
      \footnote{In future work (\cref{h:observer-script-withdraw-zero-trick}), Hydrozoa may implement support on L2 for observer scripts (CIP-112 \citep{DiSarroCIP112ObserveScript2024}) as a better alternative for dApps that currently rely on withdraw-zero staking validators for performance optimization.}
    \item[No pre-Conway features.] Hydrozoa does not need to maintain backward compatibility with pre-Conway eras.
      L2 utxos cannot use bootstrap addresses or Shelley addresses with Plutus versions older than Plutus V3.

      Public key hash credentials, native scripts, and Plutus scripts at or above V3 are allowed.
    \item[No datum hashes in outputs.] Hydrozoa requires all utxo datums to be inline, which avoids the need to index the datum-hash to datum map from transactions' datum witnesses.
    \item[No minting or burning of tokens.]
      Hydrozoa's L1 scripts are not authorized to mint or burn arbitrary user tokens onchain, so enforcing such events is impossible when settling the L2 state on L1.
      Thus, L2 transactions cannot mint or burn tokens.%
      \footnote{In future work (\cref{h:future-work-minting}), Hydrozoa may implement support for minting and burning on L2.}
    \item[No collateral on L2.]
      Phase-2 scripts (e.g. Plutus) can potentially have arbitrary resource costs (up to the Cardano's ExUnits limits for txs), but block producers do not receive normal fees from transaction that fail phase-2 validation.
      To cover the resource costs incurred for such failing phase-2 validations and to prevent denial-of-service attacks using such transactions, Cardano requires collateral to be provided in all transactions containing phase-2 scripts.

      However, in Hydrozoa's unanimous consensus protocol, any peer can deny service simply by stopping its responses to L2 messages.
      Furthermore, a peer can disconnect from the source of too many transactions failing phase-2 validation, mitigating the incurred resource costs.
      Therefore, Hydrozoa does not need collateral inputs, fees, and outputs on L2.
    \item[POSIX and time-validity intervals.]
      Hydrozoa uses POSIX for time-validity interval bounds on L2.
      An L2 transaction or withdrawal can be included in a block's valid events if the block's creation time falls within the transaction or withdrawal's time-validity interval.
\end{description}

\section{Utxo set}%
\label{h:l2-utxo-set}%

Conceptually, a utxo set is a set of transaction outputs tagged with unique references to their origins.
In practice, we equivalently represent a utxo set as a map from output reference to output:
\begin{equation*}
\begin{split}
  \T{UtxoSet^{L2}} &\coloneq \T{Map} \; \T{OutputRef} \; \T{Output^{L2}} \\
    &\coloneq \left\{
      (k_i :: \T{OutputRef}, v_i :: \T{Output^{L2}})
      \;\middle|\;
      \forall i \neq j.\; k_i \neq k_j
    \right\}
\end{split}
\end{equation*}

An output reference is a tuple that uniquely identifies an output by a hash of the transaction that created it and its index among that transaction's outputs:
\begin{equation*}
  \T{OutputRef} \coloneq \left\{
    \begin{array}{lll}
      \T{id} &::& \T{TxId} \\
      \T{index} &::& \T{UInt}
    \end{array} \right\}
\end{equation*}

An output is a tuple describing a bundle of tokens, data, and a script that have been placed at an address in the ledger:
\begin{equation*}
\begin{split}
  \T{Output^{L2}} &\coloneq \left\{
  \begin{array}{lll}
    \T{addr} &::& \T{Address^{L2}} \\
      \T{value} &::& \T{Value} \\
      \T{datum} &::& \T{Datum^{L2}} \\
      \T{script} &::& \T{Maybe} \; \T{Script^{L2}}
  \end{array} \right\} \\\\
  \T{Address^{L2}} &\coloneq \left\{
    \begin{array}{l}
      \T{Network^{L2}} \\
      \T{PaymentCredential} \\
      \T{StakeReference}
    \end{array}\right\} \\\\
  \T{Datum^{L2}} &\coloneq \T{NoDatum} \mid \T{Datum} \; \T{Data} \\\\
  \T{Script^{L2}} \coloneq\;& \T{TimelockScript} \; \T{Timelock} \\
                        \mid\;& \T{PlutusScript} \; \T{PlutusVersion^{L2}} \; \T{PlutusBinary} \\\\
  \T{PlutusVersion^{L2}} \coloneq\;& \T{PlutusV3}
\end{split}
\end{equation*}

\section{Transaction}%
\label{h:ledger-transaction}%

An L2 transaction spends one or more utxos in \code{utxosActive}, replacing them with zero or more new utxos.
The rules for Hydrozoa's L2 transactions are the same as Cardano's ledger rules for L1 transactions but with the additions listed in \cref{h:l2-ledger-rules}.

Conveniently, we can avoid forking Cardano's ledger rules library to implement these additional rules in Hydrozoa.
Instead, we express them as requirements for specific transaction fields to be omitted or restricted to a narrower selection of values.

The hydrozoa node software should reject (with an error message) L2 transactions that violate these requirements, while peers should ignore \code{ReqTx} messages (\cref{h:l2-consensus-transaction}) that broadcast such transactions.

In the following type definition, we use these notational conventions:
\begin{itemize}
  \item Some fields are already optional in Cardano transactions, with suitable defaults provided when omitted.
  We indicate these by prefixing their field types with a question mark (\code{?}).
  \item We indicate fields that must be omitted in Hydrozoa L2 transactions by prefixing their field names with an empty-set symbol (\code{$\varnothing$}).
\end{itemize}
\begingroup
\allowdisplaybreaks
\begin{align*}
    \T{Tx^{L2}} \coloneq\;& \left\{
    \begin{array}{lll}
      \T{body} &::& \T{TxBody^{L2}} \\
        \T{wits} &::& \T{TxWits^{L2}} \\
        \T{is\_valid} &::& \T{Bool} \\
        \T{auxiliary\_data} &::& \quad?\;\T{TxMetadata}
    \end{array} \right\} \\\\
    \T{TxBody^{L2}} \coloneq\;& \left\{
    \begin{array}{lll}
      \T{spend\_inputs} &::& \T{Set \; OutputRef} \\
        \varnothing\;\T{collateral\_inputs} &::& \quad?\;\T{Set \; OutputRef} \\
        \T{reference\_inputs} &::& \quad?\;\T{Set \; OutputRef} \\
        \T{outputs} &::& \T{[Output^{L2}]} \\
        \varnothing\;\T{collateral\_return} &::& \quad?\;\T{Output^{L2}} \\
        \varnothing\;\T{total\_collateral} &::& \quad?\;\T{Coin} \\
        \varnothing\;\T{certificates} &::& \quad?\;\T{[ Set \; Certificate ]} \\
        \varnothing\;\T{withdrawals} &::& \quad?\;\T{Map \; RewardAccount \; Coin} \\
        \T{fee} &::& \T{Coin} \\
        \T{validity\_interval} &::& \quad?\;\T{ValidityInterval} \\
        \T{required\_signer\_hashes} &::& \quad?\;\T{[VKeyCredential]} \\
        \varnothing\;\T{mint} &::& \quad?\;\T{Value} \\
        \T{script\_integrity\_hash} &::& \quad?\;\T{ScriptIntegrityHash} \\
        \T{auxiliary\_data\_hash} &::& \quad?\;\T{AuxiliaryDataHash} \\
        \T{network\_id} &::& \quad?\;\T{Network^{L2}} \\
        \varnothing\;\T{voting\_procedures} &::& \quad?\;\T{VotingProcedures} \\
        \varnothing\;\T{proposal\_procedures} &::& \quad?\;\T{Set \; ProposalProcedure} \\
        \varnothing\;\T{current\_treasury\_value} &::& \quad?\;\T{Coin} \\
        \varnothing\;\T{treasury\_donation} &::& \quad?\;\T{Coin}
    \end{array} \right\} \\\\
    \T{TxWits^{L2}} \coloneq\;& \left\{
    \begin{array}{lll}
      \T{addr\_tx\_wits} &::& \quad?\;\T{Set \; (VKey, Signature,  VKeyHash)} \\
        \varnothing\;\T{boot\_addr\_tx\_wits} &::& \quad?\;\T{Set \; BootstrapWitness} \\
        \T{script\_tx\_wits} &::& \quad?\;\T{Map \; ScriptHash \; Script^{L2}} \\
        \varnothing\;\T{data\_tx\_wits} &::& \quad?\;\T{TxDats} \\
        \T{redeemer\_tx\_wits} &::& \quad?\;\T{Redeemers}
    \end{array} \right\}
\end{align*}
\endgroup

Cardano's ledger rules do not allow omitting the \code{collateral\_inputs}, \code{collateral\_return}, and \code{total\_collateral} fields in transactions containing Plutus scripts.
Thus, we \emph{deactivate} the following Cardano ledger rules when validating L2 transactions:
\begin{itemize}
  \item Minimum collateral amount
  \item Non-empty collateral inputs (if Plutus scripts are used)
  \item Collateral balancing:
    \begin{equation*}
      \sum \T{lovelace} (\T{collateral\_inputs}) =
      \T{lovelace} (\T{collateral\_return}) + \T{total\_collateral}
    \end{equation*}
\end{itemize}

\section{Withdrawal}%
\label{h:ledger-withdrawal}%

A withdrawal is an L2 ledger event that spends one or more utxos in \code{utxosActive} but does not output any utxos.
The L2 ledger rules for withdrawals are similar to transactions but with the following modifications:
\begin{description}
  \item[No outputs.] Withdrawals cannot create utxos in \code{utxosActive}.
  \item[No fees.] Withdrawals do not pay transaction fees.
  \item[No metadata.] Withdrawals cannot contain transaction metadata.
\end{description}

We can express the rules for L2 withdrawals (\codeMathTt{Tx^{L2W}}) as further restrictions on the fields of L2 transactions  (\codeMathTt{Tx^{L2}}):%
\footnote{Here, ``omitting'' the \inlineColored{outputs} and \inlineColored{fee} fields means setting them to an empty list and zero, respectively.}
\begingroup
\allowdisplaybreaks
\begin{align*}
  \T{Tx^{L2W}} \coloneq\;& \left\{
    \begin{array}{lll}
      \T{body} &::& \T{TxBody^{L2W}} \\
        \T{wits} &::& \T{TxWits^{L2}} \\
        \T{is\_valid} &::& \T{Bool} \\
        \varnothing\;\T{auxiliary\_data} &::& \quad?\;\T{TxMetadata}
    \end{array} \right\} \subseteq \T{Tx^{L2}} \\\\
    \T{TxBody^{L2W}} \coloneq\;& \left\{
    \begin{array}{lll}
      \T{spend\_inputs} &::& \T{Set \; OutputRef} \\
        \varnothing\;\T{collateral\_inputs} &::& \quad?\;\T{Set \; OutputRef} \\
        \T{reference\_inputs} &::& \quad?\;\T{Set \; OutputRef} \\
        \varnothing\;\T{outputs} &::& \T{[Output^{L2}]} \\
        \varnothing\;\T{collateral\_return} &::& \quad?\;\T{Output^{L2}} \\
        \varnothing\;\T{total\_collateral} &::& \quad?\;\T{Coin} \\
        \varnothing\;\T{certificates} &::& \quad?\;\T{[ Set \; Certificate ]} \\
        \varnothing\;\T{withdrawals} &::& \quad?\;\T{Map \; RewardAccount Coin} \\
        \varnothing\;\T{fee} &::& \T{Coin} \\
        \T{validity\_interval} &::& \quad?\;\T{ValidityInterval} \\
        \T{required\_signer\_hashes} &::& \quad?\;\T{[VKeyCredential]} \\
        \varnothing\;\T{mint} &::& \quad?\;\T{Value} \\
        \T{script\_integrity\_hash} &::& \quad?\;\T{ScriptIntegrityHash} \\
        \T{auxiliary\_data\_hash} &::& \quad?\;\T{AuxiliaryDataHash} \\
        \T{network\_id} &::& \quad?\;\T{Network^{L2}} \\
        \varnothing\;\T{voting\_procedures} &::& \quad?\;\T{VotingProcedures} \\
        \varnothing\;\T{proposal\_procedures} &::& \quad?\;\T{Set \; ProposalProcedure} \\
        \varnothing\;\T{current\_treasury\_value} &::& \quad?\;\T{Coin} \\
        \varnothing\;\T{treasury\_donation} &::& \quad?\;\T{Coin}
    \end{array} \right\}
\end{align*}
\endgroup

Cardano's ledger rules do not allow the \code{fee} to be zero. Thus, we \emph{deactivate} the following Cardano ledger rules when validating L2 withdrawals, in addition to the rules deactivated for L2 transactions:
\begin{itemize}
  \item Minimum fee amount
  \item Transaction balancing:
    \begin{equation*}
      \sum \T{value} (\T{inputs}) + \T{mint} =
      \sum \T{value} (\T{outputs}) + \T{fee}
    \end{equation*}
\end{itemize}

\section{Genesis}%
\label{h:ledger-genesis}%

L2 genesis is a ledger event automatically inserted into the ledger's history immediately following the affirmed events of a confirmed major L2 block.
It adds new utxos to the \code{utxosActive} set without spending any utxos:
\begin{equation*}
  \T{Genesis^{L2}} \coloneq \left\{
  \begin{array}{lll}
    \T{utxosAdded} &::& \T{UtxoSet^{L2}}
  \end{array}\right\}
\end{equation*}
These L2 utxos uniquely correspond to the L1 \code{absorbedDeposits} of the major block (\cref{h:l2-blocks}):
\begin{description}
  \item[output reference:]~
    \begin{description}
      \item[id.] The Blake2b-256 hash (\codeMath{\mathcal{H}_{32}}) of \code{absorbedDeposits}.
      Incidentally, this is also the genesis event's unique ID (\code{TxId}).
      \item[index.] The corresponding L1 deposit's positional index in \code{depositsAbsorbed}.
    \end{description}
  \item[output:]~
    \begin{description}
      \item[addr.] The \code{address} field in the corresponding L1 deposit's datum.
      \item[value.] The value of the corresponding L1 deposit.
      \item[datum.] The \code{datum} field in the corresponding L1 deposit's datum.
      \item[script.] Empty.
    \end{description}
  \end{description}

\end{document}
