\documentclass[../hydrozoa.tex]{subfiles}
\graphicspath{{\subfix{../images/}}}
\begin{document}

\chapter{L2 consensus protocol}%
\label{h:l2-consensus-protocol}%

L2 consensus between a Hydrozoa head's peers is achieved via a peer-to-peer protocol that:
\begin{itemize}
  \item Broadcasts L2 transaction and withdrawal requests among the peers.
  \item Facilitates agreement between the peers on a growing common sequence of L2 ledger events that are grouped into blocks.
  \item Promptly settles the L1 effects of a confirmed block, if it affirms any L2 withdrawals or absorbs any L1 deposits.
  \item Ensures that no funds get stranded as unabsorbed L1 deposits. If a deposit will not be absorbed into the head, it can always be refunded to the depositor.
  \item Ensures that no funds get stranded in the head. If the peers stop responding to each other, they can always retrieve their funds on L1 according to the latest block matching the L1 treasury's major version.
\end{itemize}

Peers take round-robin turns to create L2 blocks---every $i^\mathtt{th}$ block must be created by the  $i^\mathtt{th}$ peer.

\section{Setup}%
\label{h:l2-consensus-setup}%

\subsection{Parameters~~~(\codeHeading{ReqParams} / \codeHeading{AckParams})}%
\label{h:l2-consensus-parameters}%

% TODO parameters for L2 consensus

% TODO ReqParams
% Alice: Let's open a head! Here are my proposed parameters.

% TODO AckParams
% Bob: Acknowledged. Here's my pubkey.
% Charlie: Acknowledged. Here's my pubkey.

\subsection{Initialization~~~(\codeHeading{ReqInit} / \codeHeading{AckInit})}%
\label{h:l2-consensus-intitialization}%

% TODO ReqInit
% Alice: Please sign this head initialization transaction, which uses a native script parametrized by Alice, Bob, and Charlie's provided pubkeys.

% TODO AckInit
% Bob: here's my signature.
% Charlie: here's my signature.
% (Everyone): (Submits the multisigned init transaction to Cardano)

\section{Ledger requests}%
\label{h:l2-consensus-ledger}%

Any peer can broadcast a request to the other peers, asking them to witness a new L2 transaction or withdrawal.
The peers do not explicitly acknowledge the request, but a future block must affirm the L2 transaction or withdrawal if it is valid relative to the block creator's peer state.

\subsection{Transaction~~~(\codeHeading{ReqTx})}%
\label{h:l2-consensus-transaction}%

A request from one peer to the other peers to witness a new L2 transaction:
\begin{equation*}
  \T{ReqTx} \coloneq \bigr\{\; \T{tx} :: \T{Tx^{L2}} \;\bigr\}
\end{equation*}

Upon receiving a (\codeMathTt{\T{ReqTx} \; \T{tx}}) at time \code{timeReceived}, a peer must append the following L2 event to the \code{eventsL2} table in the peer's state (\cref{h:l2-peer-state}):
\begin{equation*}
  \left\{
  \begin{array}{lll}
    \T{timeReceived} &\coloneq& \T{timeReceived} \\
    \T{eventId} &\coloneq& \T{tx.txId} \\
    \T{eventType} &\coloneq& \T{Tranasction} \\
    \T{event} &\coloneq& \T{tx} \\
    \T{blockNum} &\coloneq& \varnothing
  \end{array}\right\}
\end{equation*}
The sender must do the same, immediately upon sending the request.

No acknowledgment response is required.
Peers other than the next block creator should not attempt to validate this L2 transaction, at this time.

\subsection{Withdrawal~~~(\codeHeading{ReqWithdrawal})}%
\label{h:l2-consensus-withdrawal}%

A request from one peer to the other peers to witness a new L2 withdrawal:
\begin{equation*}
  \T{ReqWithdrawal} \coloneq \bigr\{\; \T{withdrawal} :: \T{Tx^{L2W}} \;\bigr\}
\end{equation*}

Upon receiving a (\codeMathTt{\T{ReqWithdrawal} \; \T{w}}) at time \code{timeReceived}, a peer must append the following L2 event to the \code{eventsL2} table in the peer's state (\cref{h:l2-peer-state}):
\begin{equation*}
  \left\{
  \begin{array}{lll}
    \T{timeReceived} &\coloneq& \T{timeReceived} \\
    \T{eventId} &\coloneq& \T{w.txId} \\
    \T{eventType} &\coloneq& \T{Withdrawal} \\
    \T{event} &\coloneq& \T{w} \\
    \T{blockNum} &\coloneq& \varnothing
  \end{array}\right\}
\end{equation*}
The sender must do the same, immediately upon sending the request.

No acknowledgment response is required.
Peers other than the next block creator should not attempt to validate this L2 withdrawal, at this time.

\section{Consensus on blocks}%
\label{h:l2-consensus-on-blocks}%

If there are unconfirmed L2 events, and no new block is awaiting confirmation from the peers, then the next block must be created and sent to the peers for confirmation.
The $i^\mathtt{th}$ peer has the exclusive right to create every $i^\mathtt{th}$ block: 
\begin{equation*}
  \T{rightfulBlockCreator} \; \T{peersN} \; \T{blockNum} \coloneq
    \T{mod} \; \T{blockNum} \; \T{peersN}
\end{equation*}

To create the a block, apply the block creation procedure (\cref{h:l2-block-creation}) to the block creator's peer state.
Depending on the type of the new block, the block creator must broadcast either a \code{ReqMinor}, \code{ReqMajor}, or \code{ReqFinal} request to the other peers.

Each recipient must validate the new block relative to the recipient's peer state. If the new block is valid, the recipient must broadcast a corresponding acknowledgment response (\code{AckMinor}, \code{AckMajor1}, or \code{AckFinal1}) to the other peers.

The block creator must broadcast a corresponding acknowledgment response for the new block to the other peers, immediately upon broadcasting the new block.

When the new block is major or final, acknowledgment proceeds in two rounds.%
\footnote{The two rounds are needed for atomic block confirmation.
  Major and final blocks have multiple effects that require signatures from peers.
  To prevent any peer from obtaining all signatures for one effect of a block without broadcasting the peer's signature for other effects, the first round collects signatures for all effects except for the L1 settlement transaction.
  None of the other effects can happen without the L1 settlement transaction.
  Once all signatures for the other effects are collected in the first round, signatures for the L1 settlement can be collected in the second round.
  }
When a peer has received the first acknowledgment response (\code{AckMajor1} or \code{AckFinal1}) from all peers, the peer must broadcast a corresponding second acknowledgment response (\code{AckMajor2} or \code{AckFinal2}).

A peer must consider a block and all its affirmed events to be confirmed when all of the block's acknowledgments have been received.

\subsection{Minor block~~~(\codeHeading{ReqMinor} / \codeHeading{AckMinor})}%
\label{h:l2-consensus-minor-block}%

% TODO

\subsection{Major block~~~(\codeHeading{ReqMajor} / \codeHeading{AckMajor1} / \codeHeading{AckMajor2})}%
\label{h:l2-consensus-major-block}%

% TODO

\subsection{Final block~~~(\codeHeading {ReqFinal} / \codeHeading{AckFinal1} / \codeHeading{AckFinal2})}%
\label{h:l2-consensus-final-block}%

% TODO

\section{Consensus on refunds}%
\label{h:l2-consensus-on-refunds}%


\subsection{Post-dated refund~~~(\codeHeading {ReqRefundLater} / \codeHeading{AckRefundLater})}%
\label{h:l2-consensus-post-dated-refund}%

% TODO

\subsection{Immediate refund~~~(\codeHeading{ReqRefundNow} / \codeHeading{AckRefundNow})}%
\label{h:l2-consensus-immediate-refund}%

% TODO

\end{document}
