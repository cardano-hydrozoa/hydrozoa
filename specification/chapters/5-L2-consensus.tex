\documentclass[../hydrozoa.tex]{subfiles}
\graphicspath{{\subfix{../images/}}}
\begin{document}

\chapter{L2 consensus protocol}%
\label{h:l2-consensus-protocol}%

L2 consensus between a Hydrozoa head's peers is achieved via a peer-to-peer protocol that:
\begin{itemize}
  \item Broadcasts L2 transaction and withdrawal requests among the peers.
  \item Facilitates agreement between the peers on a growing common sequence of L2 ledger events that are grouped into blocks.
  \item Promptly settles the L1 effects of a confirmed block, if it affirms any L2 withdrawals or absorbs any L1 deposits.
  \item Ensures that no funds get stranded as unabsorbed L1 deposits.
  If a deposit will not be absorbed into the head, it can always be refunded to the depositor.
  \item Ensures that no funds get stranded in the head.
  If the peers stop responding to each other, they can always retrieve their funds on L1 according to the latest block matching the L1 treasury's major version.
\end{itemize}

Peers take round-robin turns to create L2 blocks---every $i^\mathtt{th}$ block must be created by the  $i^\mathtt{th}$ peer.

While participating in the L2 consensus protocol, each peer maintains this local state:%
\footnote{This type is defined in \cref{h:l2-peer-state} and repeated here.}
\begin{equation*}
  \peerStateTypeName{} \coloneq \peerStateTypeBody{}
\end{equation*}

\section{Assumptions}%
\label{h:l2-consensus-assumptions}%

The L2 consensus protocol relies on these assumptions:
\begin{enumerate}
  \item The peers have pre-established pairwise communication channels with each other.
  \item Every network message received from a peer is checked for authentication.
    An implementation of this specification needs to find a suitable method to authenticate either the communication channels or the individual messages.
  \item Every peer is promptly and correctly notified about all relevant L1 transactions, and L1 consensus finality and liveness properties always hold.
  \item The lifecycle of a Hydrozoa head never crosses any L1 protocol update boundaries.%
    \footnote{These boundaries are often announced in advance, before the hard-fork combinator event, so Hydrozoa peers should be capable of avoiding them by finalizing existing heads and delaying initialization of new heads.}
\end{enumerate}

\section{Setup}%
\label{h:l2-consensus-setup}%

During setup, the initiator proposes the L2 consensus parameters for the head, collects the peers' verification keys, and drafts the L1 initialization transaction for the head.
The other peers respond to the initiator's requests, broadcasting their responses to all peers.

When all the peers agree on this information, they submit the multi-signed initialization transaction.

\subsection{Parameters~~~(\codeHeading{ReqParams} / \codeHeading{AckParams})}%
\label{h:l2-consensus-parameters}%

To start the setup process, the initiator must broadcast a request for the other peers to adopt the initiator's proposed L2 consensus parameters:%
\footnote{The L2 consensus parameters are defined in \cref{h:l2-peer-state} and repeated here.}
\begin{equation*}
\begin{split}
  \T{ReqParams} &\coloneq \{\; \T{params} :: \paramsConsensusTypeName{} \;\bigr\} \\
  \paramsConsensusTypeName{} &\coloneq \paramsConsensusTypeBody{}
\end{split}
\end{equation*}

On receiving a \code{ReqParams}, if the receiving peer accepts the proposed parameters, the peer must update the \code{params} in its state to the proposed parameters and broadcast an acknowledgment that contains the peer's verification key:
\begin{equation*}
  \T{AckParams} \coloneq \;\bigr\{\; \T{vkey} :: \T{VerificationKey} \;\bigr\}
\end{equation*}
The initiator do the same, immediately after sending the \code{ReqParams} request.

On receiving a \code{AckParams}, the receiving peer must update the \code{peers} list in its peer state to include the received verification key and its Blake2b-256 (\codeMath{\mathcal{H}_{32}}) hash.
This list must have unique verification keys and must be sorted in ascending order of public key hash.
Furthermore, it must always include the peer's own verification key and public key hash.

\subsection{Initialization~~~(\codeHeading{ReqInit} / \codeHeading{AckInit})}%
\label{h:l2-consensus-intitialization}%

On receiving all parameter acknowledgments from the peers, the initiator must draft an L1 initialization transaction (\cref{h:l1-multisig-initialization}) for the head, using the peers' public key hashes, and broadcast it to the peers:
\begin{equation*}
  \T{ReqInit} \coloneq \;\bigr\{\; \T{initTx} :: \T{Tx^{L1}} \;\bigr\}
\end{equation*}

On receiving a \code{ReqInit}, the receiving peer must verify that the initialization transaction meets the conditions in \cref{h:l1-multisig-initialization} and uses a native script (\cref{h:l1-multisig-regime}) parametrized by all the peers' public key hashes, corresponding to the \code{peers} list its peer state.
If it is valid, the peer must sign it with the peer's verification key and broadcast it an acknowledgment:
\begin{equation*}
  \T{AckInit} \coloneq \left\{
  \begin{array}{lll}
    \T{vkeyHash} &::& \T{PublicKeyHash} \\
    \T{signature} &::& \T{Signature}
  \end{array}\right\}
\end{equation*}
The initiator must do the same, immediately on sending the \code{ReqInit} request.

On receiving an \code{AckInit}, the receiving peer must verify that \code{vkeyHash} is a required signatory and \code{signature} is a valid signature of the intialization transaction by the verification key corresponding to \code{vkeyHash}.
If so, the peer must attach the signature to the initialization transaction.

Each peer must submit the initialization transaction when all required signatures are attached.

\section{Ledger requests}%
\label{h:l2-consensus-ledger}%

Any peer can broadcast a request to the other peers, asking them to witness a new L2 transaction or withdrawal.
The peers do not explicitly acknowledge the request, but a future block must affirm the L2 transaction or withdrawal if it is valid relative to the block creator's peer state.

\subsection{Transaction~~~(\codeHeading{ReqTx})}%
\label{h:l2-consensus-transaction}%

To submit a new L2 transaction to the head, the submitting peer must broadcast this request:
\begin{equation*}
  \T{ReqTx} \coloneq \;\bigr\{\; \T{tx} :: \T{Tx^{L2}} \;\bigr\}
\end{equation*}

On receiving a \code{ReqTx} at time \code{timeReceived}, the receiving peer must append the following L2 event to the \code{eventsL2} table in  in its peer state (\cref{h:l2-peer-state}):
\begin{equation*}
  \left\{
  \begin{array}{lll}
    \T{timeReceived} &\coloneq& \T{timeReceived} \\
    \T{eventId} &\coloneq& \T{tx.txId} \\
    \T{eventType} &\coloneq& \T{Tranasction} \\
    \T{event} &\coloneq& \T{tx} \\
    \T{blockNum} &\coloneq& \varnothing
  \end{array}\right\}
\end{equation*}
The submitter must do the same, immediately on sending the request.

No acknowledgment is required.
Peers other than the next block creator should not attempt to validate this L2 transaction, at this time.

\subsection{Withdrawal~~~(\codeHeading{ReqWithdrawal})}%
\label{h:l2-consensus-withdrawal}%

To submit a new L2 withdrawal to the head, the submitting peer must broadcast this request:
\begin{equation*}
  \T{ReqWithdrawal} \coloneq \;\bigr\{\; \T{withdrawal} :: \T{Tx^{L2W}} \;\bigr\}
\end{equation*}

On receiving a \code{ReqWithdrawal} at time \code{timeReceived}, the receiving peer must append the following L2 event to the \code{eventsL2} table in its peer state (\cref{h:l2-peer-state}):
\begin{equation*}
  \left\{
  \begin{array}{lll}
    \T{timeReceived} &\coloneq& \T{timeReceived} \\
    \T{eventId} &\coloneq& \T{w.txId} \\
    \T{eventType} &\coloneq& \T{Withdrawal} \\
    \T{event} &\coloneq& \T{w} \\
    \T{blockNum} &\coloneq& \varnothing
  \end{array}\right\}
\end{equation*}
The submitter must do the same, immediately on sending the request.

No acknowledgment is required.
Peers other than the next block creator should not attempt to validate this L2 withdrawal, at this time.

\section{Consensus on blocks}%
\label{h:l2-consensus-on-blocks}%

If there are unconfirmed L2 events, and no new block is awaiting confirmation from the peers, then the next block must be created and sent to the peers for confirmation.
The $i^\mathtt{th}$ peer has the exclusive right to create every $i^\mathtt{th}$ block: 
\begin{equation*}
  \T{rightfulBlockCreator} \; \T{peersN} \; \T{blockNum} \coloneq
    \T{mod} \; \T{blockNum} \; \T{peersN}
\end{equation*}

To create a new block, apply the block creation procedure (\cref{h:l2-block-creation}) to the block creator's peer state.
Depending on the new block's type, the block creator must broadcast it in either a \code{ReqMinor}, \code{ReqMajor}, or \code{ReqFinal} request to the other peers.

Each recipient must validate the new block relative to the recipient's peer state.
If the new block is valid, the recipient must broadcast a corresponding acknowledgment (\code{AckMinor}, \code{AckMajor1}, or \code{AckFinal1}) to the other peers.

The block creator must broadcast a corresponding acknowledgment for the new block to the other peers, immediately on broadcasting the new block.

When the new block is major or final, acknowledgment proceeds in two rounds.%
\footnote{The two rounds are needed for atomic block confirmation.
  Major and final blocks have multiple effects that require signatures from peers.
  To prevent any peer from obtaining all signatures for one effect of a block without broadcasting the peer's signature for other effects, the first round collects signatures for all effects except for the L1 settlement transaction.
  None of the other effects can happen without the L1 settlement transaction.
  Once all signatures for the other effects are collected in the first round, signatures for the L1 settlement can be collected in the second round.
  }
When a peer has received the first acknowledgment (\code{AckMajor1} or \code{AckFinal1}) from all peers, the peer must broadcast a corresponding second acknowledgment (\code{AckMajor2} or \code{AckFinal2}).

When a peer receives all acknowledgments for a block, the peer must consider the block and all its affirmed events to be confirmed.
The peer must update its peer state as described in \cref{h:l2-block-effects}.

\subsection{Minor block~~~(\codeHeading{ReqMinor} / \codeHeading{AckMinor})}%
\label{h:l2-consensus-minor-block}%

A block creator's broadcast of a new minor block contains the block:
\begin{equation*}
  \T{ReqMinor} \coloneq \;\bigr\{\; \T{block} :: \T{Block} \;\bigr\}
\end{equation*}

A peer's acknowledgment of a minor block identifies the block by block number and contains the recipient's signature of the new block's header:
\begin{equation*}
  \T{AckMinor} \coloneq \left\{
  \begin{array}{lll}
    \T{blockNum} &::& \T{UInt} \\
    \T{vkey} &::& \T{UInt} \\
    \T{signature} &::& \T{Signature}
  \end{array}\right\}
\end{equation*}

\subsection{Major block~~~(\codeHeading{ReqMajor} / \codeHeading{AckMajor1} / \codeHeading{AckMajor2})}%
\label{h:l2-consensus-major-block}%

A block creator's broadcast of a new major block contains the block;
\begin{equation*}
  \T{ReqMajor} \coloneq \;\bigr\{\; \T{block} :: \T{Block} \;\bigr\}
\end{equation*}

A peer's first acknowledgment of a major block identifies the block by block number and contains the recipient's signatures for the block's rollout and post-dated transition transactions:
\begin{equation*}
  \T{AckMajor1} \coloneq \left\{
  \begin{array}{lll}
    \T{vkey} &::& \T{VerificationKey} \\
    \T{rollouts} &::& [\T{Signature}] \\
    \T{postDatedTransition} &::& \T{Signature}
  \end{array}\right\}
\end{equation*}

A peer's second acknowledgment of a major block identifies the block by block number and contains the recipient's signature for the block's settlement transaction:
\begin{equation*}
  \T{AckMajor2} \coloneq \left\{
  \begin{array}{lll}
    \T{vkey} &::& \T{VerificationKey} \\
    \T{settlement} &::& \T{Signature}
  \end{array}\right\}
\end{equation*}

\subsection{Final block~~~(\codeHeading {ReqFinal} / \codeHeading{AckFinal1} / \codeHeading{AckFinal2})}%
\label{h:l2-consensus-final-block}%

A block creator's broadcast of a new major block contains the block;
\begin{equation*}
  \T{ReqFinal} \coloneq \;\bigr\{\; \T{block} :: \T{Block} \;\bigr\}
\end{equation*}

A peer's first acknowledgment of a major block identifies the block by block number and contains the recipient's signatures for the block's rollout transactions:
\begin{equation*}
  \T{AckFinal1} \coloneq \left\{
  \begin{array}{lll}
    \T{vkey} &::& \T{VerificationKey} \\
    \T{rollouts} &::& [\T{Signature}]
  \end{array}\right\}
\end{equation*}

A peer's second acknowledgment of a major block identifies the block by block number and contains the recipient's signature for the block's finalization transaction:
\begin{equation*}
  \T{AckFinal2} \coloneq \left\{
  \begin{array}{lll}
    \T{vkey} &::& \T{VerificationKey} \\
    \T{finalization} &::& \T{Signature}
  \end{array}\right\}
\end{equation*}

\section{Consensus on refunds}%
\label{h:l2-consensus-on-refunds}%

To ensure that deposited funds never get stranded, every depositor must obtain a multi-signed post-dated refund transaction from the head's peers before sending the L1 deposit utxo to the head's native script address.

Similarly, the peers may sometimes determine that an L1 deposit will not be absorbed into the head---e.g., insufficient time before deposit deadline, disruption in L2 block consensus.
In those cases, the peers should attempt to multi-sign and submit an immediate refund transaction for the deposit, as a courtesy to the depositor.

\subsection{Post-dated refund~~~(\codeHeading {ReqRefundLater} / \codeHeading{AckRefundLater})}%
\label{h:l2-consensus-post-dated-refund}%

A request for a post-dated refund of an L1 deposit must contain the unsigned transaction that will create the L1 deposit:
\begin{equation*}
  \T{ReqRefundLater} \coloneq \;\{\; \T{depositTx} :: \T{Tx^{L1}} \;\}
\end{equation*}

On receiving a \code{ReqRefundLater}, the receiving peer must:
\begin{enumerate}
  \item Verify that the deposit transaction creates a valid L1 deposit (\cref{h:l1-multisig-deposit}).
  \item Construct an L1 refund transaction that spends the L1 deposit and sends its funds to the \code{refundAddress} and \code{refundDatum}, with time validity interval starting at the \code{deadline}.
  \item Broadcast the peer's verification key and signature for the post-dated refund transaction:
    \begin{equation*}
      \T{AckRefundLater} \coloneq \left\{
      \begin{array}{lll}
        \T{refundTxId} &::& \T{TxId} \\
        \T{vkey} &::& \T{VerificationKey} \\
        \T{signature} &::& \T{Signature}
      \end{array}\right\}
    \end{equation*}
\end{enumerate}

When the depositor obtains all the peers' signatures for the post-dated refund transaction, the depositor can cache it and submit the deposit transaction.

\subsection{Immediate refund~~~(\codeHeading{ReqRefundNow} / \codeHeading{AckRefundNow})}%
\label{h:l2-consensus-immediate-refund}%

A request for an immediate refund must contain the list of L1 deposit utxos to be refunded:
\begin{equation*}
  \T{ReqRefundNow} \coloneq \;\{\; \T{deposits} :: [\T{OutputReference}] \;\}
\end{equation*}

On receiving a \code{ReqRefundNow}, the receiving peer must:
\begin{enumerate}
  \item Verify that each of the L1 deposit utxos exists on L1 and that no more than \code{depositMarginExpiry} time duration is left until the deposit's \code{deadline}.
  \item Construct an L1 refund transaction that spends the L1 deposits and sends corresponding outputs, according to the deposits' \code{refundAddress} and \code{refundDatum} fields.
    This transaction's time-validity interval is unbounded.
  \item Broadcast the peer's verification key and signature for the immediate refund transaction:
    \begin{equation*}
      \T{AckRefundNow} \coloneq \left\{
      \begin{array}{lll}
        \T{refundTxId} &::& \T{TxId} \\
        \T{vkey} &::& \T{VerificationKey} \\
        \T{signature} &::& \T{Signature}
      \end{array}\right\}
    \end{equation*}
\end{enumerate}

When a peer obtains all the peers' signatures for the immediate refund transaction, the peer must submit it.

\end{document}
