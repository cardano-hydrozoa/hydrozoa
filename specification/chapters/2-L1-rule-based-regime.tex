\documentclass[../hydrozoa.tex]{subfiles}
\graphicspath{{\subfix{../images/}}}
\begin{document}

\input{\subfix{0-constants}}

\chapter{L1 rule-based regime}%
\label{h:l1-rule-based-regime}%

If the peers stop responding to each other in the L2 consensus protocol, the L1~rule-based~regime ensures that the peers can still withdraw their funds from the Hydrozoa head's treasury.
It is less cost-efficient and more complex than the multisig regime, but it serves as a fallback when L2 consensus breaks down.

The transition from the multisig regime to the rule-based regime pins the L1 treasury's major version, disqualifying any L2 blocks with lesser or greater major versions.%
\footnote{Blocks with lesser major versions are disqualified because they assume a total treasury balance of funds that may no longer exist due to deposits and withdrawals adding/removing funds from the treasury.
  Blocks with greater major versions are disqualified because the transition to the rule-based regime precludes their settlement transactions from being confirmed on L1 (utxo contention over the multisig treasury).}
The rule-based regime then proceeds via the following suite of Plutus scripts:
\begin{itemize}
  \item The dispute resolution scripts (\cref{h:l1-rule-based-dispute-resolution}) arbitrate the peers' dispute about the latest confirmed L2 block matching the L1 treasury's major version.
  \item The treasury spending validator (\cref{h:l1-rule-based-treasury}) awaits resolution and then manages withdrawals from the head's treasury based on the resolved L2 block.
\end{itemize}

The rule-based regime does not manage any unabsorbed deposits left over from the multisig regime.
Depositors can retrieve them with post-dated refund transactions (\cref{h:l1-multisig-refund}) from the multisig regime.

\section{Utxo state}%
\label{h:l1-rule-based-utxo-state}%

\subsection{Treasury state}

In the rule-based regime, the head's treasury utxo is at the head's Plutus-based treasury address (\cref{h:l1-rule-based-treasury}).
It holds the head's beacon token and all treasury funds from the multisig regime (\cref{h:l1-multisig-utxo-state}).

The treasury utxo must have the following datum type, which defines its unresolved and resolved states in the rule-based regime:
\begin{equation*}
\begin{split}
  \T{RuleBasedTreasuryDatum} \coloneq&\;
    \T{Resolved} \; \T{ResolvedDatum} \\\mid&\;
    \T{Unresolved} \; \T{UnresolvedDatum} \\
  \T{ResolvedDatum} \coloneq&\;\left\{
    \begin{array}{lll}
      \T{headMp} &::& \T{PolicyId} \\
      \T{utxosActive}  &::& \mathcal{RH}_{32} \; \T{UtxoSet^{L2}} \\
      \T{version} &::& (\T{UInt}, \T{UInt}) \\
      \T{params} &::& \mathcal{H}_{32} \; \T{ParamsHead}
    \end{array}\right\} \\
  \T{UnresolvedDatum} \coloneq&\;\left\{
    \begin{array}{lll}
      \T{headMp} &::& \T{PolicyId} \\
      \T{disputeId} &::& \T{AssetName} \\
      \T{peers} &::& \T{[VerificationKey]} \\
      \T{peersN} &::& \T{UInt} \\
      \T{deadlineVoting} &::& \T{PosixTime} \\
      \T{versionMajor} &::& \T{UInt} \\
      \T{params} &::& \mathcal{H}_{32} \; \T{ParamsHead}
    \end{array}\right\}
\end{split}
\end{equation*}

The resolved treasury's datum is the same as the multisig treasury's datum, with these exceptions:
\begin{description}
  \item[head MP.] The policy ID of the head's native script minting policy.
  \item[version.] The full version of the resolved L2 block, which includes both the major and minor version numbers. This replaces the \code{versionMajor} field of the multisig treasury.
\end{description}

The unresolved treasury's datum fields are interpreted as follows:
\begin{description}
  \item[head MP.] The policy ID of the head's native script minting policy.
  \item[dispute ID.] The unique identifier used as the asset name of the dispute tokens (see below).
  \item[peers.] The peers' public keys.
  \item[n peers.] The number of peers in the head.
  \item[voting deadline.] A POSIX time before which votes can be cast in the dispute.
  \item[major version.] The treasury's major version, pinned by the transition to the rule-based regime.
  \item[params.] A 32-byte Blake2b-256 hash (\codeMath{\mathcal{H}_{32}}) of the head's L2 consensus parameters (\cref{h:l2-consensus-protocol}).
\end{description}

\subsection{Dispute state}

Besides the treasury utxo, the head's state in the rule-based regime also includes the vote utxos for the dispute.
Each of these utxos is at the head's Plutus-based dispute address and holds one or more of the dispute's fungible tokens:
\begin{itemize}
  \item The policy ID corresponds to the head's native script (in minting policy form) (\cref{h:l1-multisig-regime}).
  \item The first four bytes of the asset name are equal to the CIP-67
    \citep{AlessandroKonradThomasVellekoopCIP67AssetName2022}
    prefix for class label \headDisputeToken{}, which corresponds to Hydrozoa head dispute tokens.%
    \footnote{We chose \headDisputeToken{} because it spells ``DISP'' (short for dispute) on a phone dial pad.}
  \item The last 28 bytes of the asset name are equal to the Blake2b-224 hash (\codeMath{\mathcal{H}_{28}}) of the multisig treasury utxo spent in the head's transition to the rule-based regime (\cref{h:l1-rule-based-transition}).
\end{itemize}

Each vote utxo has this datum type:
\begin{equation*}
\begin{split}
  \T{VoteDatum} &\coloneq \left\{
    \begin{array}{lll}
      \T{key}  &::& \T{UInt} \\
      \T{link} &::& \T{UInt} \\
      \T{peer} &::& \T{Maybe} \; \T{PubKeyHash} \\
      \T{voteStatus} &::& \T{VoteStatus}
    \end{array}\right\}\\
  \T{VoteStatus} &\coloneq \T{NoVote} \mid \T{Abstain} \mid \T{Vote} \; \T{VoteDetails} \\
  \T{VoteDetails} &\coloneq \left\{
    \begin{array}{lll}
      \T{utxosActive} &::& \mathcal{RH}_{32} \; \T{UtxoSet^{L2}} \\
      \T{versionMinor} &::& \T{UInt}
    \end{array}\right\}
\end{split}
\end{equation*}

These fields are interpreted as follows:
\begin{description}
  \item[key.] An integer that uniquely identifies the vote utxo.
  \item[link.] An integer that uniquely references another vote utxo by its \code{key}.
  \item[peer.] The public-key hash of the peer (if any) who can cast the vote in this utxo. It is only empty for the default vote utxo (with \code{key} 0), which contains a vote that is automatically cast at the transition to the rule-based regime (\cref{h:l1-rule-based-transition}).
  \item[vote status.] Either no vote, a vote to abstain, or a vote for a minor L2 block.
  \item[active utxos.] A 32-byte Merkle root hash (\codeMath{\mathcal{RH}_{32}}) of the L2 ledger's active utxo set, as of the block voted by the peer.
  \item[minor version.] The minor version of the minor L2 block voted by the peer. This block's major version must match the major version that the treasury had when it transitioned to the rule-based regime.
\end{description}

Collectively, the vote utxos constitute a linked list data structure on L1.

\section{Transition to rule-based regime}%
\label{h:l1-rule-based-transition}%

The L2 consensus protocol pairs every multisig treasury utxo that it creates with a post-dated transaction that transitions the treasury to the rule-based regime (\cref{h:l2-consensus-protocol}).

While L2 consensus holds, the peers store these post-dated transactions without submitting.
However, if the multisig utxo is not spent by the time its corresponding post-dated transaction becomes valid, then L2 consensus is assumed to have stalled.
In that case, every peer should trigger the transition to the rule-based regime by submitting the transaction to Cardano.

The transaction body is as follows:
\begin{enumerate}
  \item Spend the multisig treasury utxo (\code{multisigTreasury}).
    Let \code{mt} be its datum.
  \item Let \code{headMp} be the head's native script minting policy.
  \item Let \code{deadlineVoting} be the voting deadline for this dispute, as defined by the L2 consensus for this transition to the rule-based regime.
  \item Let \code{disputeId} be the CIP-67 prefix for 3477, concatenated with the Blake2b-224 hash (\codeMath{\mathcal{H}_{28}}) of the \code{multisigTreasury} output reference.
  \item Let \code{peers} be a list of the peers' public keys and \code{peersN} be its length.%
    \footnote{The indices of the \code{peers} list (and all other lists in this specification) start from \inlineColored{1}.}
  \item Mint (\code{peersN} + 1) dispute tokens of (\code{headMp}, \code{disputeId}).
  \item Create the rule-based treasury utxo (\code{ruleBasedTreasury}), containing the funds and beacon token from \code{multisigTreasury} and this datum:
    \begin{equation*}
    \begin{split}
      \T{transitionTreasuryDatum} \coloneq \T{Unresolved} \;\left\{
        \begin{array}{lll}
          \T{headMp} &=& \T{headMp} \\
          \T{disputeId} &=& \T{disputeId} \\
          \T{peers} &=& \T{peers} \\
          \T{peersN} &=& \T{peersN} \\
          \T{deadlineVoting} &=& \T{deadlineVoting} \\
          \T{versionMajor} &=& \T{mt.majorVersion} \\
          \T{params} &=& \T{mt.params}
        \end{array}\right\}
    \end{split}
    \end{equation*}
  \item Create the default vote utxo, containing a vote for the major L2 block's active utxo set:
    \begin{equation*}
      \T{defaultVoteDatum} \coloneq \left\{
      \begin{array}{lll}
        \T{key} &=& 0 \\
        \T{link} &=& 0 < \T{peersN} \;?\; 1 : 0 \\
        \T{peer} &=& \T{Nothing} \\
        \T{voteStatus} &=& \T{Vote} \left\{
        \begin{array}{lll}
            \T{utxosActive} &=& \T{mt.utxosActive} \\
            \T{versionMinor} &=& 0
        \end{array}\right\}
      \end{array}\right\}
    \end{equation*}
  \item For \codeMath{i} iterated from 1 to \code{peersN}, create a vote utxo with one dispute token and this datum:%
    \begin{equation*}
      \T{initVoteDatum} \; i \coloneq \left\{
        \begin{array}{lll}
          \T{key}  &=& i \\
          \T{link} &=& i < \T{peersN} \;?\; (i + 1) : 0 \\
          \T{peer} &=& \T{Just} \; (\mathcal{H}_{32} \; \T{peers}[i]) \\
          \T{voteStatus} &=& \T{NoVote}
        \end{array}\right\}\\
    \end{equation*}
\end{enumerate}

\section{Dispute resolution}%
\label{h:l1-rule-based-dispute-resolution}%

During the voting period of a dispute, each peer gets one opportunity to cast their vote on the latest minor L2 block confirmed by the L2 consensus protocol before it stalled.
After all votes are cast or the voting period ends, the dispute state can be resolved to a single vote utxo containing the latest block among the votes.

\subsection{Spending validator for voting}

Voting is handled by the Plutus-based spending validator of the dispute address.
Redeemers:
\begin{description}
  \item[Vote.] Replace the input's no-vote status with an abstention or a valid vote for a block. Conditions:
    \begin{itemize}
      \item Verify the vote inputs:
        \begin{enumerate}
          \item Let \code{voteOutref} be the output reference of the input being spent.
          \item Let \code{voteInput} be the input being spent.
            There must not be any other spent input matching \code{voteOutref} on tx hash.
          \item The \code{voteStatus} field of \code{voteInput} must be empty.
          \item The transaction must be signed by the \code{peer} field of \code{voteInput}, if it is non-empty.
          \item Let (\code{headMp}, \code{disputeId}) be the minting policy and asset name of the only non-ADA tokens in \code{voteInput}.
        \end{enumerate}
      \item Verify the treasury reference input:
        \begin{enumerate}[resume]
          \item Let \code{treasury} be the only reference input matching \code{voteOutref} on tx hash.
          \item A head beacon token of \code{headMp} and CIP-67 prefix \headBeaconToken{} must be in \code{treasury}.
          \item \code{headMp} and \code{disputeId} must match the corresponding fields of the \code{Unresolved} datum in \code{treasury}.
          \item The transaction's time-validity upper bound must not exceed the \code{deadlineVoting} field of \code{treasury}.
        \end{enumerate}
      \item Verify the redeemer:
        \begin{enumerate}[resume]
          \item Let \code{voteRedeemer} be an \emph{optional} redeemer argument with this type:
            \begin{equation*}
            \begin{split}
              \T{VoteRedeemer} &\coloneq \left\{
                \begin{array}{lll}
                  \T{blockHeader} &::& \T{BlockHeader^{L2}} \\
                  \T{multisig} &::& [\T{Signature}]
                \end{array}\right\}\\
              \T{BlockHeader^{L2}} &\coloneq \left\{
              \begin{array}{lll}
                \T{blockNum} &::& \T{UInt} \\
                \T{blockType} &::& \T{BlockType^{L2}} \\
                \T{creationTime} &::& \T{PosixTime} \\
                \T{versionMajor} &::& \T{UInt} \\
                \T{versionMinor} &::& \T{UInt} \\
                \T{utxosActive} &::& \mathcal{RH}_{32} \; \T{UtxoSet^{L2}}
              \end{array}\right\}
            \end{split}
            \end{equation*}
          \item If \code{voteRedeemer} is provided, then both of these must hold:
            \begin{enumerate}
              \item The \code{multisig} field of \code{voteRedeemer} must have signatures of the \code{blockHeader} field of \code{voteRedeemer} for all the public keys in the \code{peers} field of \code{treasury}.
              \item The \code{versionMajor} field must match between \code{treasury} and \code{voteRedeemer}.
            \end{enumerate}
        \end{enumerate}
      \item Verify the vote output:
        \begin{enumerate}[resume]
          \item Let \code{voteOutput} be an output with the same number of (\code{headMp}, \code{disputeId}) tokens and same address as \code{voteInput}. It must not hold any other non-ADA tokens.
          \item If \code{voteRedeemer} is provided, the \code{voteStatus} field of \code{voteOutput} must be a \code{Vote} matching \code{voteRedeemer} on the \code{utxosActive} and \code{versionMinor} fields.
            Otherwise, it must be \code{Abstain}.
          \item All other fields of \code{voteInput} and \code{voteOutput} must match.
        \end{enumerate}
    \end{itemize}
  \item[Tally.] Refer validation to the staking validator of the dispute address.
    Conditions:
    \begin{enumerate}
      \item Let \code{tallyValidator} be the script hash of the staking validator for tallying, provided as a static parameter to the spending validator for voting.
      \item Let \code{voteOutref} be the output reference of the input being spent.
      \item The transaction's redeemers list must show \code{tallyValidator} being invoked to withdraw-zero with a redeemer that mentions \code{voteOutref} in one of its fields.
    \end{enumerate}
  \item[Resolve.] The sole remaining vote utxo can be spent when the treasury utxo is spent.
    Conditions:
    \begin{enumerate}
      \item Let \code{voteInput} be the input being spent.
      \item Let (\code{headMp}, \code{disputeId}) be the minting policy and asset name of the only non-ADA tokens in \code{voteInput}.
      \item Let \code{treasury} be a spent input that holds a head beacon token of \code{headMp} and CIP-67 prefix \headBeaconToken{}.
      \item \code{headMp} and \code{disputeId} must match the corresponding fields of the \code{Unresolved} datum in \code{treasury}.
\end{enumerate}
\end{description}

\subsection{Staking validator for tallying}

Tallying is handled by the Plutus-based staking validator of the dispute address, via the withdraw-zero script purpose.
Redeemers:
\begin{description}
  \item[Tally.] Combine two vote utxos into one, selecting the highest vote.
    Conditions:
    \begin{itemize}
      \item Verify the vote inputs:
        \begin{enumerate}
          \item Let \code{tallyRedeemer} be a redeemer argument with this type:
            \begin{equation*}
              \T{TallyRedeemer} \coloneq \left\{
              \begin{array}{lll}
                \T{continuingOutref} &::& \T{OutputRef} \\
                \T{removedOutref} &::& \T{OutputRef}
              \end{array}\right\}
            \end{equation*}
          \item Let \code{continuingInput} and \code{removedInput} be spent inputs with the respective output references of \code{tallyRedeemer}.
          \item \code{continuingInput} and \code{removedInput} must match on address.
          \item \code{continuingInput} and \code{removedInput} must have non-ADA tokens of only one asset class, which must match between them and satisfy both of the following:
            \begin{enumerate}
              \item The minting policy (\code{headMp}) must be a native script.
              \item The asset name (\code{disputeId}) must have the CIP-67 prefix \headDisputeToken{}.
            \end{enumerate}
          \item The \code{key} field of \code{removedInput} must be greater than the \code{key} field and equal to the \code{link} field of \code{continuingInput}.
          \item There must be no other spent inputs from the same address as \code{continuingInput} or holding any tokens of \code{headMp}.
        \end{enumerate}
      \item Verify the treasury reference input:
        \begin{enumerate}[resume]
          \item If the \code{voteStatus} of either \code{continuingInput} or \code{removedInput} is \code{NoVote}, all of the following must be satisfied:
            \begin{enumerate}
              \item Let \code{treasury} be a reference input holding the head beacon token of \code{headMp} and CIP-67 prefix \headBeaconToken{}.
              \item \code{headMp} and \code{disputeId} must match the corresponding fields of the \code{Unresolved} datum in \code{treasury}.
              \item The \code{deadlineVoting} field of \code{treasury} must not exceed the transaction's time-validity lower bound.
            \end{enumerate}
        \end{enumerate}
      \item Verify the vote output:
        \begin{enumerate}[resume]
          \item Let \code{continuingOutput} be an output with the same address and the sum of all tokens (including ADA) in \code{continuingInput} and \code{removedInput}.
          \item The \code{voteStatus} field of \code{continuingOutput} must match the highest \code{voteStatus} of \code{continuingInput} and \code{removedInput}, totally ordered as follows:
            \begin{equation*}
            \begin{split}
              &\forall a, b \in \T{VoteDetails}:\\
              &\quad a \T{.minorVersion} \leq b \T{.minorVersion} \implies\\
              &\qquad \T{NoVote} < \T{Abstain} < \T{Vote} \; a \leq \T{Vote} \; b  
            \end{split}
            \end{equation*}
          \item The \code{link} field of \code{removedInput} and \code{continuingOutput} must match.
          \item All other fields of \code{continuingInput} and \code{continuingOutput} must match.
        \end{enumerate}
    \end{itemize}
\end{description}

\section{Treasury}%
\label{h:l1-rule-based-treasury}%

The treasury spending validator awaits the dispute's resolution and adopts the active utxo set of the resolved L2 block.
Peers can then withdraw funds from the treasury by creating utxos on L1 and showing Merkle proofs of equivalent utxos being removed from the treasury's active utxo set.
Redeemers:
\begin{description}
  \item[Resolve.] Resolve the treasury's state based on the sole remaining vote utxo.
    Conditions:
    \begin{enumerate}
      \item Let \code{treasuryInput} be the input being spent.
      \item Let \code{headMp}, \code{disputeId}, \code{peersN}, and \code{versionMajor} be the corresponding fields of \code{treasuryInput}.
      \item Let \code{voteInput} be a spent input with (\code{peersN} + 1) tokens of (\code{headMp}, \code{disputeId}) and no other non-ADA tokens.
      \item Let \code{treasuryOutput} be an output with the same address as \code{treasuryInput} and the sum of all tokens (including ADA) in \code{treasuryInput} and \code{voteInput}.
      \item Let \code{voteStatus} be the corresponding field of \code{voteInput}.
      \item If \code{voteStatus} is \code{Abstain}:
        \begin{enumerate}
          \item The \code{version} field of \code{treasuryOutput} must match (\code{versionMajor}, 0).
          \item \code{treasuryInput} and \code{treasuryOutput} must match on \code{utxosActive}.
        \end{enumerate}
      \item If \code{voteStatus} is \code{Vote}:
        \begin{enumerate}
          \item Let \code{versionMinor} be the corresponding field in \code{voteStatus}.
          \item The \code{version} field of \code{treasuryOutput} must match (\code{versionMajor}, \code{versionMinor}).
          \item \code{voteStatus} and \code{treasuryOutput} must match on \code{utxosActive}.
        \end{enumerate}
      \item \code{treasuryInput} and \code{treasuryOutput} must match on all other fields.
    \end{enumerate}
  \item[Withdraw.] Withdraw treasury funds and update its active utxo set accordingly.
    Conditions:
    \begin{enumerate}
      \item Let \code{treasuryInput} be the input being spent.
      \item Let \code{headMp} be the corresponding field in \code{treasuryInput}.
      \item Let \code{headId} be the asset name of the only \code{headMp} token in \code{treasuryInput} with CIP-67 prefix \headBeaconToken{}.
      \item Let \code{treasuryOutput} be the output holding the (\code{headMp}, \code{headId}) beacon token.
      \item Let \code{withdrawals} be a list of triples. In each triple:
        \begin{itemize}
          \item The \code{key} is an output reference of a utxo in the active utxo set corresponding to the \code{utxosActive} field of \code{treasuryInput}.
          \item The \code{value} is an output of this transaction.
          \item The \code{proof} is a Merkle inclusion proof.
        \end{itemize}
      \item Starting with the \code{utxosActive} field in \code{treasuryInput}, folding over \code{withdrawals} with the Merkle library's \code{delete} function must produce a result matching the \code{utxosActive} field of \code{treasuryOutput}:%
        \footnote{The details of this function depend on the Merkle library that will be used by the Hydrozoa implementation. For example, see the \inlineColored{delete} function in the Merkle Patricia Forestry library \citep{MatthiasBenkortMerklePatriciaForestry2024}.}
        \begin{equation*}
        \begin{split}
          &\T{validWithdrawals} \; \T{in} \; \T{ws} \; \T{out} \coloneq
            (\T{foldl} \; \T{delete} \; \T{in} \; \T{ws} = \T{out}) \\
          &\T{delete} ::
            \mathcal{RH}_{32} \; \T{UtxoSet^{L2}} ->
            (\T{OutputRef}, \T{Output^{L2}}, \T{Proof}) ->
            \mathcal{RH}_{32} \; \T{UtxoSet^{L2}}
        \end{split}
        \end{equation*}
      \item \code{treasuryInput} must hold the sum of all tokens in \code{treasuryOutput} and the outputs of \code{withdrawals}.
    \end{enumerate}
  \item[Deinit.] De-initialize the head, removing all traces of it.
    Conditions:%
    \footnote{This redeemer does not require the treasury's active utxo set to be empty, but it does implicitly require the transaction to be multi-signed by all peers to burn the \inlineColored{headMp} tokens.
      Thus, the peers can also use this redeemer to override the treasury's spending validator with their multi-signature.}
    \begin{enumerate}
      \item Let \code{treasury} be the input being spent.
      \item Let \code{headMp} be the corresponding field in \code{treasury}.
      \item All \code{headMp} tokens in \code{treasury} with CIP-67 prefixes of \headDisputeToken{} and \headBeaconToken{} must be burned.
    \end{enumerate}
\end{description}

\end{document}
