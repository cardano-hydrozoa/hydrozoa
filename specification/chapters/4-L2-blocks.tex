\documentclass[../hydrozoa.tex]{subfiles}
\graphicspath{{\subfix{../images/}}}
\begin{document}

\chapter{L2 blocks}%
\label{h:l2-blocks}%

It would be inefficient for a head's peers to reach consensus on each L2 ledger event individually.
Instead, they reach consensus on an entire block of L2 ledger events at every round of the Hydrozoa L2 consensus protocol.

All L2 blocks have the following type:
\begin{equation*}
\begin{split}
  \T{Block^{L2}} \coloneq&\; \left\{
    \begin{array}{lll}
      \T{header} &::& \T{BlockHeader^{L2}} \\
      \T{body} &::& \T{BlockBody^{L2}}
    \end{array}\right\} \\
  \T{BlockHeader^{L2}} \coloneq&\; \left\{
    \begin{array}{lll}
      \T{blockNum} &::& \T{UInt} \\
      \T{blockType} &::& \T{BlockType^{L2}} \\
      \T{versionMajor} &::& \T{UInt} \\
      \T{versionMinor} &::& \T{UInt} \\
      \T{utxosActive} &::& \mathcal{RH}_{32} \; \T{UtxoSet^{L2}}
    \end{array}\right\} \\
  \T{BlockType^{L2}} &\coloneq
    \T{Minor} \mid
    \T{Major} \mid
    \T{Final} \mid
    \T{Initial} \\
  \T{BlockBody^{L2}} &\coloneq \left\{
  \begin{array}{lll}
    \T{eventsValid} &::&
      \T{Sequence} \; (\T{EventType^{L2}}, \T{TxId}) \\
    \T{eventsInvalid} &::&
      \T{Map} \; \T{TxId} \; \T{EventType^{L2}} \\
    \T{depositsAbsorbed} &::& \T{Sequence} \; \T{OutputRef}
  \end{array}\right\} \\
  \T{EventType^{L2}} &\coloneq \T{Transaction} \mid \T{Withdrawal}
\end{split}
\end{equation*}

Each block affirms a sequence of L2 ledger events, rejects a separate set of L2 ledger events, and absorbs a set of L1 deposits.
The block's header contains a Merkle root hash of the active utxo set that results from applying the affirmed L2 ledger events to the latest confirmed L2 ledger state.

Depending on the types of L2 ledger events affirmed by the block and the peers' intent to continue operating the head, there are four kinds of L2 blocks in Hydrozoa:
\begin{description}
  \item[Minor block.] A block that neither affirms any L2 withdrawals nor absorbs any L1 deposits.
    It has no effect on the head's L1 utxo state in the multisig regime and only affects the L2 ledger state's active utxo set.
    They may affect the L1 utxo state in the rule-based regime.
  \item[Major block.] A block that affirms some L2 withdrawals or absorbs some L1 deposits.
    It implicitly affirms an L2 settlement event corresponding to its affirmed L2 withdrawals and absorbed L1 deposits.
    It immediately affects the head's L1 utxo state as soon as it is confirmed by the peers, via the L1 settlement and rollout transactions that mirror its L2 settlement event (\cref{h:ledger-settlement}).
  \item[Final block.] A major block with which the peers finalize the head, but which must not absorb any L1 deposits.
    It implicitly affirms withdrawals for the entire active utxo set that remains after its sequence of affirmed L2 ledger events.
    Its L2 settlement event results in an empty L2 ledger state.
  \item[Initial block.] A block with which the peers initialize the head.
    It has an empty block body and results in an empty L2 ledger state.
    It is a notional block that is presumed confirmed when the peers initialize the head.
    As such, it does not need to be created, validated, or explicitly confirmed by the peers.
\end{description}

Blocks are numbered consecutively by \code{blockNum}, and they are versioned by \code{versionMajor} and \code{versionMinor} numbers:
\begin{itemize}
  \item The initial block has both major and minor versions set to zero.
  \item A minor block keeps its predecessor's major version and increments the minor version.
  \item A major block increments the major version and resets the minor version to zero.
  \item The final block increments the major version and resets the minor version to zero.
\end{itemize}

\section{Block validation environment}%
\label{h:l2-block-validation-environment}%

Every peer requires access to an up-to-date source of the following information:
\begin{equation*}
\begin{split}
  \T{BlockValidationEnv} &\coloneq \left\{
    \begin{array}{lll}
      \T{timeCurrent} &::& \T{PosixTime} \\
      \T{paramsBlockValidation} &::& \T{ParamsBlockValidtion} \\
      \T{stateL1} &::& \T{MultisigHeadState^{L1}} \\
      \T{stateL2} &::& \T{LedgerState^{L2}} \\
      \T{blocksConfirmedL2} &::& \T{Sequence} \; \T{Block^{L2}} \\
      \T{eventsSeenL2} &::& \T{Table} \; \T{Event^{L2}}
    \end{array}\right\} \\
  \T{ParamsBlockValidation} &\coloneq \left\{
    \begin{array}{lll}
      \T{depositMarginMaturity} &::& \T{UDiffTime} \\
      \T{depositMarginExpiry} &::& \T{UDiffTime}
    \end{array}\right\} \\
  \T{Event^{L2}} &\coloneq \left\{
    \begin{array}{lll}
      \T{timeReceived} &::& \T{PosixTime} \\
      \T{eventId} &::& \T{TxId} \\
      \T{eventType} &::& \T{EventType^{L2}} \\
      \T{event} &::& \T{Tx^{L2}} \\
      \T{blockNum} &::& \T{Maybe} \; \T{UInt}
    \end{array}\right\} \\
\end{split}
\end{equation*}

These fields are interpreted as follows:
\begin{description}
  \item[current time.] The current POSIX time, synchronized via the NTP protocol \citep{MillsEtAlNetworkTimeProtocol2010}.
  \item[block validation params.] The head's block validation parameters:
    \begin{description}
      \item[deposit maturity margin.] After an L1 deposit is created on L1, the peers must wait for this non-negative time duration before attempting to absorb it into the head's treasury.
      \item[deposit expiry margin.] If less than this non-negative time duration is left before an L1 deposit's deadline, then the peers must not attempt to absorb it into the head's treasury.
    \end{description}
  \item[L1 state.] The head's L1 utxo state in the multisig regime (\cref{h:l1-multisig-utxo-state}).%
    \footnote{The head's L1 state is only needed to create/validate major and final blocks, which cannot be done when the head is in the L1 rule-based regime.}
  \item[L2 state.] The head's L2 ledger state (\cref{h:l2-ledger}) as of the latest confirmed block.
  \item[confirmed L2 blocks.] The sequence of L2 blocks confirmed in the L2 consensus protocol.
  \item[seen L2 events.] The table of L2 events witnessed by the peer,%
    \footnote{It's up to the L2 consensus protocol whether this table stores all L2 events ever witnessed by the peer, or just the recent events up to a time-based or block-based cutoff.}
    which must be unique on \code{eventId} and sorted in the ascending order of \code{timeReceived}:
    \begin{description}
      \item[time received.] The POSIX time at which the peer received the event.
      \item[event ID.] The transaction ID corresponding to the event's effect on the L2 active utxo set.
      \item[event type.] The L2 event is a transaction, withdrawal, or settlement.
      \item[event.] The transaction representing the event's effect on the L2 active utxo set.
      \item[block number.] The block number of the block (if any) that includes the event, regardless of whether the block is confirmed.
    \end{description}
\end{description}

With the above information, a peer can create and determine the validation status of a block:
\begin{equation*}
  \T{ValidationStatus} \coloneq
    \T{Valid} \mid
    \T{NotYetValid} \mid
    \T{Invalid}
\end{equation*}
These statuses are interpreted as follows:
\begin{description}
  \item[Valid.] The block is valid and should be confirmed by the peer.
  \item[Not yet valid.] The peer has detected an error that could be resolved if the peer waits and tries again---e.g., the peer may soon receive an L2 event that was included in the block.
  \item[Invalid.] The peer has detected an error that cannot be resolved by waiting.
\end{description}

\section{Block creation}%
\label{h:l2-block-creation}%

The initial block is predefined and identical for all Hydrozoa heads.
All other blocks are created by a common procedure that branches based on whether the peers want to finalize the head, the peers want to force a major block, or the procedure has detected any absorbable L1 deposits or unconfirmed L2 withdrawals in the block validation environment.

Given the block validation environment, create a block as follows:
% TODO Can/should this procedure be rewritten in a non-imperative programming style?
\begin{enumerate}
  \item Initialize variables and arguments (immutable by default):
    \begin{enumerate}
      \item Let \code{finalize} be an argument indicating whether a final block should be created.
      \item Let \code{forceMajor} be an argument indicating whether a major block should be created.
      \item Let \code{previousBlock} be the latest element in \code{blocksConfirmedL2}.
      \item Let \code{block} be a mutable variable initialized to an empty \codeMathTt{Block^{L2}}.
      \item Let \code{localStateL2} be a mutable variable initialized to \code{stateL2}.
      \item Let \code{seenValidWithdrawals} be a mutable variable initialized to \code{False}.
      \item Let \code{major} be a mutable variable initialized to \code{False}.
    \end{enumerate}
  \item For each L2 event \code{x} in \code{eventsSeenL2} in ascending order of \code{timeReceived}:
      \begin{enumerate}
        \item If \code{x.eventType} is a \code{Transaction}, try to apply \code{x.event} to \code{localStateL2.utxosActive}.
          If it is valid according to Hydrozoa's ledger rules for L2 transactions (\codeMathTt{Tx^{L2}}):
          \begin{enumerate}
            \item Append (\code{x.eventId}, \code{x.eventType}) to \code{block.eventsValid}.
            \item Update \code{localStateL2.utxosActive} to the result of this transition.
          \end{enumerate}
        \item If \code{x.eventType} is a \code{Withdrawal}, try to apply \code{x.event} to \code{localStateL2.utxosActive}.
          If it is valid according to Hydrozoa's ledger rules for L2 withdrawals (\codeMathTt{Tx^{L2W}}):
          \begin{enumerate}
            \item Append (\code{x.eventId}, \code{x.eventType}) to \code{block.eventsValid}.
            \item Update \code{localStateL2.utxosActive} to the result of this transition.
            \item Insert the spent inputs of \code{x.event} into \code{localStateL2.utxosWithdrawn}.
            \item Update \code{seenValidWithdrawals} to \code{True}.
          \end{enumerate}
        \item Otherwise, insert (\code{x.eventId}, \code{x.eventType}) into \code{block.eventsInvalid}.
      \end{enumerate}
  \item For each deposit \code{d} in \code{stateL1.depositUtxos}, insert \code{d} into \code{block.depositsAbsorbed} if:
    \begin{equation*}
      \T{d.deadline} + \T{depositMarginMaturity} \leq
      \T{timeCurrent} <
      \T{d.deadline} - \T{depositMarginExpiry}
    \end{equation*}
  \item Update \code{major} to \code{True} if any of the following hold:
    \begin{enumerate}
      \item \code{seenValidWithdrawals} is \code{True}.
      \item \code{block.depositsAbsorbed} is non-empty.
      \item \code{forceMajor} is \code{True}.
      \item \code{finalize} is \code{True}.
    \end{enumerate}
  \item Set block header:
    \begin{enumerate}
      \item Set \code{block.blockNum} to (\code{previousBlock.blockNum} + 1).
      \item Set \code{block.utxosActive} to the Merkle root hash of \code{localStateL2.utxosActive}.
      \item If \code{major} is \code{True}:
        \begin{enumerate}
          \item Set \code{block.versionMajor} to (\code{previousBlock.versionMajor} + 1).
          \item Set \code{block.versionMinor} to zero.
        \end{enumerate}
      \item Otherwise:
        \begin{enumerate}
          \item Set \code{block.versionMajor} to \code{previousBlock.versionMajor}.
          \item Set \code{block.versionMinor} to (\code{previousBlock.versionMinor} + 1).
        \end{enumerate}
    \end{enumerate}
\end{enumerate}

% TODO fix

% TODO major blocks created in two cases:
% - Accumulation of unabsorbed L1 deposits or unsettled L2 withdrawals
% - Within safety margin of L1 treasury transition boundary (consensus protocol parameter)

\section{Block validation}%
\label{h:l2-block-validation}%

% TODO

\subsection{Minor block}%
\label{h:l2-block-validation-minor}%

% TODO

\subsection{Major block}%
\label{h:l2-block-validation-major}%

% TODO

\subsection{Final block}%
\label{h:l2-block-validation-final}%

% TODO

\section{Block application}%
\label{h:l2-block-application}%

% TODO

\subsection{Minor block}%
\label{h:l2-block-application-minor}%

% TODO

\subsection{Major block}%
\label{h:l2-block-application-major}%

% TODO

\subsection{Final block}%
\label{h:l2-block-application-final}%

% TODO

\end{document}
