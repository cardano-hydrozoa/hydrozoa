\documentclass[../hydrozoa.tex]{subfiles}
\graphicspath{{\subfix{../images/}}}
\begin{document}

\section{Shrinking linked list}%
\label{h:shrinking-linked-list}

% TODO Don't prematurely generalize this data structure.
% Let's first describe it in the specific instance of how it's used in hydrozoa, and defer generalization to later.

A shrinking linked list is a variant of the key-ordered linked list onchain data structure, where the list is initialized with all the nodes that it will ever have.
Afterward, list nodes can only be removed or have their app data modified in-place.
The list can only be de-initialized when it is empty.

This data structure is useful in situations where the full list must be initialized by a collective decision, but subsequent transitions do not need collective authorization.
For example, the collective initializes a shrinking list with a multi-signed transaction (without executing any plutus scripts), and then individuals transition list nodes according to plutus-enforced rules.

% TODO root node of empty list holds all the tokens until they can be burned.

% TODO non-root node of singleton list is removed when resolving the dispute.

\end{document}
