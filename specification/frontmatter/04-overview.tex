\documentclass[../hydrozoa.tex]{subfiles}
\graphicspath{{\subfix{../images/}}}
\begin{document}

\chapter*{Overview}%
\label{h:overview-introduction}%
\addcontentsline{toc}{chapter}{Overview}%

Hydrozoa is a dynamic, near-isomorphic multi-party state channel protocol:
\begin{description}
  \item[Multi-party state channel.] Hydrozoa allows a group of peers to transact with one another directly outside of the Cardano blockchain (on L2), significantly minimizing their activity on the Cardano blockchain (on L1).
    The peers enjoy high throughput, instant finality, and low costs while transacting with each other in this way.
  \item[Near-isomorphic.] The peers' transactions on L2 follow the same ledger rules as Cardano L1 to the extent possible.%
    \footnote{Unfortunately, full equivalence is impossible because the result of the L2 transactions ultimately needs to be settled on L1 by the peers, and Cardano's ledger rules do not authorize the peers to perform certain actions (\cref{h:l2-ledger-rules}).}
  \item[Dynamic.] If the peers unanimously agree, Hydrozoa allows them to add new peers or remove existing peers from the group,%
    \footnote{Adding and removing peers is not currently specified in this document, but will be specified and implemented in the final milestone of the Catalyst Fund 13 Hydrozoa project \citep{FlerovskyCatalystMilestonesHydrozoa2024} (\cref{h:future-work-add-remove-members}).}
    deposit funds from L1, and withdraw funds from L2.
\end{description}
Hydrozoa evolved out of Cardano's Hydra Head protocol \citep{NagelEtAlHydraHeadV1Specification2024}, dramatically simplifying the L1 smart contract architecture and increasing flexibility.

\section*{L1 protocol}%
\label{h:overview-l1-protocol}%
\addcontentsline{toc}{section}{L1 protocol}%

The Hydrozoa L1 protocol has two distinct operating regimes.
Every head is initialized by its peers in the \textbf{multisig} regime (\cref{h:l1-multisig-regime}), in which all funds and state are stored at a native script address controlled by the unanimous multi-signature of its peers.
Deposits into the head are created by sending utxos to this address. They are either absorbed into the head's treasury or refunded to depositors by multi-signed transactions.
Similarly, withdrawals are settled on L1 by multi-signed transactions that send utxos out of the head's treasury.
Thus, the multisig regime's L1 script defers entirely to the L2 consensus protocol between the peers, which it perceives as a black box mechanism that produces multi-signed transactions to modify the L1 state.

Ideally, a head spends its entire life cycle in the multisig regime, right up to the final transaction that withdraws all remaining funds out of the head and de-initializes it.
However, the multisig regime may stall if the peers stop multi-signing transactions before reaching finalization.

In this case, the head moves to the \textbf{rule-based} regime (\cref{h:l1-rule-based-regime}), in which unabsorbed deposits are refunded and control over the head's treasury transfers to a suite of Plutus scripts.

This regime focuses on the peers' dispute about which L2 block is the latest among those confirmed by the L2 consensus protocol and compatible with the L1 treasury state.
The Plutus scripts arbitrate the peers' dispute and manage withdrawals from the treasury based on the resolved L2 block.

\section*{L2 ledger state and blocks}%
\label{h:overview-l2-ledger-state-and-blocks}%
\addcontentsline{toc}{section}{L2 state}%

The Hydrozoa L2 ledger state (\cref{h:l2-ledger}) consists of a set of active utxos and has three types of transitions:
\begin{itemize}
  \item L2 transactions spend utxos and may produce new utxos.
  \item L2 withdrawals spend utxos but do not produce any new utxos.
  \item L2 genesis events produce new utxos but do not spend any utxos.
\end{itemize}

Among these, transactions and withdrawals are manually initiated on L2 by peers, while genesis events are automatically inserted as L1 deposits are absorbed into the head's treasury.

For efficiency, L2 events are batched into blocks (\cref{h:l2-blocks}) so that the peers can reach consensus on an entire block of L2 events at every round of the Hydrozoa L2 consensus protocol.
Each block affirms a sequence of L2 ledger events, rejects a separate set of L2 ledger events, and absorbs a set of L1 deposits.
There are three types of blocks:
\begin{itemize}
  \item A minor block only affirms L2 transactions. It does not affirm L2 withdrawals and does not absorb any L1 deposits.
  \item A major block may withdraw some utxos and absorb some deposits.
  \item A final block is intended to be the last block confirmed by the peers for the head. It withdraws the entire active utxo set and does not absorb any L1 deposits.
\end{itemize}

\section*{L2 consensus protocol}%
\label{h:overview-l2-consensus-protocol}%
\addcontentsline{toc}{section}{L2 consensus protocol}%

Hydrozoa's L2 consensus protocol (\cref{h:l2-consensus-protocol}):
\begin{itemize}
  \item Broadcasts L2 transaction and withdrawal requests among the peers.
  \item Facilitates peer agreement on a growing common sequence of L2 ledger events grouped into blocks.
  \item Promptly settles the L1 effects of a confirmed block if it affirms any L2 withdrawals or absorbs any L1 deposits.
  \item Ensures that no funds get stranded as unabsorbed L1 deposits.
  If a deposit is determined not to be absorbed into the head, it can always be refunded to the depositor.
  \item Ensures that no funds get stranded in the head's treasury.
  If the peers stop responding to each other, they can always retrieve their funds on L1 according to the latest block matching the L1 treasury's major version.
\end{itemize}

Peers take round-robin turns to create L2 blocks---every $i^\mathtt{th}$ block must be created by the  $i^\mathtt{th}$ peer.

\end{document}
