\documentclass[../hydrozoa.tex]{subfiles}
\graphicspath{{\subfix{../images/}}}
\begin{document}

\section{Multisig regime}%
\label{h:multisig-regime}%
Every Hydrozoa head uniquely corresponds to the set of its peers' public key hashes.
In the multisig regime, the head's native script simply requires signatures from all these public key hashes:%
\footnote{Here, we're using the native script constructors defined in the Cardano Mary era specification \citep{PolinaVinogradovaAndreKnispelFormalSpecificationCardano2020}.}
\begin{equation*}
\begin{split}
  \T{headNativeScript} &:: \T{Set} \; \T{KeyHash} -> \T{Script}^\T{native}_\T{v2} \\
  \T{headNativeScript} &\coloneq
    \T{AllOf} \;.\; \T{map} \; \T{Signature}
\end{split}
\end{equation*}

We could end this section here, as no other conditions are enforced by onchain scripts in the multisig regime.
Nonetheless, we will describe the expected states and transitions in this regime, which are enforced during normal operation of the Hydrozoa offchain consensus protocol (\cref{h:offchain-consensus-protocol}).%
\footnote{Of course, the peers can always manually override the offchain consensus protocol by unanimous consent.}

\subsection{Utxo state}%
\label{h:multisig-utxo-state}%
In the multisig regime, a head's entire onchain state resides in utxos at its native script address.
Among these utxos, the head's treasury utxo is uniquely identified as the one holding the head's beacon token:
\begin{enumerate}
  \item The policy ID corresponds to the head's native script (in minting policy form).
  \item The first four bytes of the asset name are equal to the CIP-67
    \citep{AlessandroKonradThomasVellekoopCIP67AssetName2022}
    prefix for class label \inlineColored{4937}, which corresponds to Hydrozoa head beacon tokens.%
    \footnote{We chose \inlineColored{4937} because it spells ``HYDR'' (short for Hydra/Hydrozoa) on a phone dial pad.}
  \item The last 28 bytes of the asset name are equal to the Blake2b-224 hash (\inlineColored{$\mathcal{H}_{28}$}) of the utxo nonce spent in the head's initialization transaction (\cref{h:multisig-init}).
\end{enumerate}
No other utxo at the head's native script address should hold any tokens that meet the policy ID and CIP-67 prefix conditions above. In other words, there must only be one head at the address.%
\footnote{It may seem redundant to use a 28-byte unique nonce in the beacon token's asset name, given that only one head is allowed per native script address at any given time.
  However, the unique nonce is helpful to distinguish between different heads that may exist at the same address across time.}

The treasury utxo must have this datum:
\begin{equation*}
  \T{MultisigTreasuryDatum} \coloneq \left\{
    \begin{array}{lll}
      \T{activeUtxos}  &::& \mathcal{RH}_{32} \; \T{UtxoSet} \\
      \T{majorVersion} &::& \T{Int} \\
      \T{params} &::& \mathcal{H}_{32} \; \T{HeadParams}
    \end{array}\right\}
\end{equation*}
These fields are interpreted as follows:
\begin{description}
  \item[active utxos.] A 32-byte Merkle root hash (\inlineColored{$\mathcal{RH}_{32}$}) of the offchain ledger's active utxo set, as of the latest major offchain block settled onchain (\cref{h:multisig-settle}).
  \item[major version.] The version number of the latest major offchain block settled onchain.
  \item[params.] A 32-byte Blake2b-256 hash (\inlineColored{$\mathcal{H}_{32}$}) of the head's offchain consensus parameters (\cref{h:offchain-consensus-protocol}).
\end{description}

Non-treasury utxos at the head's native script address are recognized as follows:
\begin{itemize}
  \item Deposits have datums with the \code{DepositDatum} type (\cref{h:multisig-deposit}).
  \item Rollout nodes are no-datum outputs of the head's settlement transactions (\cref{h:multisig-settle}).
    They are transient in nature, quickly spent in the rollout transactions that follow the settlement transactions that create them.
\end{itemize}

Unrecognized utxos are ignored by the Hydrozoa protocol.

\subsection{Initialization}%
\label{h:multisig-init}
Before initializing a Hydrozoa head, the peers should agree on the offchain consensus parameters they want to use (\cref{h:offchain-consensus-protocol}).
Initialization is achieved with a single transaction, multi-signed by all the peers, that:
\begin{itemize}
  \item Mints the head's beacon token.
  \item Spends the utxo nonce reference by hash in the last 28 bytes of the beacon token's asset name.
  \item Creates the treasury utxo at the head's native script address, placing the beacon token into it (with min-ADA) with this datum:
    \begin{equation*}
    \begin{split}
      \T{initMultisigTreasuryDatum} &:: \mathcal{H}_{32} \; \T{HeadParams} -> \T{MultisigTreasuryDatum} \\
      \T{initMultisigTreasuryDatum} &\; \T{initParams} \coloneq \left\{
        \begin{array}{lll}
          \T{activeUtxos} &=& \mathcal{RH}_{32} \; \varnothing \\
          \T{majorVersion} &=& 0 \\
          \T{params} &=& \T{initParams}
        \end{array}\right\}
    \end{split}
    \end{equation*}
\end{itemize}

\subsection{Deposit}%
\label{h:multisig-deposit}
Peers can deposit funds into a Hydrozoa head by sending utxos to its native script address, specifying their instructions in this datum type:
\begin{equation*}
  \T{DepositDatum} \coloneq \left\{
  \begin{array}{lll}
    \T{address} &::& \T{Address} \\
    \T{datum} &::& \T{Maybe \; Data} \\
    \T{deadline} &::& \T{PosixTime} \\
    \T{refundAddress} &::& \T{Address} \\
    \T{refundDatum} &::& \T{Maybe \; Data}
  \end{array}\right\}
\end{equation*}

The \code{deadline} field defines a POSIX time before which the deposit should be absorbed into the head's treasury via a settlement transaction (\cref{h:multisig-settle}).
After the deadline, the deposit should be refunded to the peer (\cref{h:multisig-refund}).

The \code{address} and \code{datum} fields define the address and datum at which the utxo in the offchain ledger should be created, if the deposit is absorbed into the head's treasury.
The new utxo in the offchain ledger should contain the same funds as the deposit utxo.

The \code{refundAddress} and \code{refundDatum} fields define the address and datum to which the deposit's funds should be sent, if the deposit is refunded.

\subsection{Refund}%
\label{h:multisig-refund}
A refund is a transaction, multi-signed by all peers, that sends a deposit utxo's funds to the deposit's \code{refundAddress} and \code{refundDatum}.
As long as the peers are following the offchain consensus protocol, they will automatically multi-sign and submit refund transactions for all deposits that are past their deadline.

However, a peer should obtain the other peers' signatures for a post-dated refund transaction before submitting the deposit transaction to Cardano.
This transaction's validity interval should start at the deposit's \code{deadline}.
The post-dated refund ensures that the peer can still retrieve the deposited funds if they haven not been absorbed by the deadline, even if the other peers stop responding.

Peers can request and receive signatures for post-dated refund transactions via the \code{ReqDeposit} and \code{AckDeposit} messages of the offchain consensus protocol (\cref{h:offchain-consensus-protocol}).
In other words, if the peer requests permission to submit the deposit (\code{ReqDeposit}), then the peers confirm their readiness to absorb it into the head and assure the refund in case they fail to do so (\code{AckDeposit}).

\subsection{Settlement and rollout}%
\label{h:multisig-settle}
A settlement is a transaction, multi-signed by all peers, that absorbs deposits into a Hydrozoa head's treasury and pays out withdrawals out of the treasury, based on a major offchain block (\cref{h:offchain-blocks}).%
\footnote{A block is ``major'' if it confirms new deposits or withdrawals that were not confirmed in the preceding blocks.}
It also updates the \code{majorVersion} and \code{activeUtxos} fields of the \code{MultisigTreasuryDatum} to the corresponding fields in the major offchain block, while keeping the \code{params} field unchanged.

The absorbed deposit utxos are spent inputs in the settlement transaction.
This means that Cardano's transaction size limit constrains the number of deposits settled per major offchain block.

On the other hand, the number of withdrawals per major offchain block is unconstrained.
Given a set of withdrawn utxos in a major offchain block, the deterministic utxo rollout algorithm (\cref{h:deterministic-utxo-rollout}) assigns them to be outputs of the settlement and rollup transactions.
Overall, they are paid out in the ascending order of their output references in the offchain ledger.

The settlement transaction greedily pays out as many withdrawals as it can, using the transaction size remaining after fitting the deposit inputs and the fixed transaction overhead.
It outputs the remaining withdrawals' aggregate funds as a single ``rollout node'' output, sent to the head's native script address without a datum.

The rollout node is spent in a rollout transaction, paying out some more withdrawals, and outputting a new rollout node with the aggregate funds of the rest of the withdrawals, if any remain.
Further rollout transactions spend the shrinking rollout node, until all the withdrawals have been paid out.

\subsection{Finalization}%
\label{h:multisig-finalize}
The finalization transaction is similar to the settlement transaction, except that:
\begin{itemize}
  \item It burns the head's beacon token.
  \item It spends the head's treasury utxo without outputting a new treasury utxo.
  \item It does not absorb any deposits.
  \item It withdraws all utxos in the final offchain block's ledger, including those in the active utxo set, using a rollout node output if necessary.
\end{itemize}

The offchain consensus protocol produces the multi-signed finalization transaction via the \code{ReqFinal} and \code{AckFinal} messages (\cref{h:offchain-consensus-protocol}).
To complement the finalization transaction, the protocol also produces multi-signed immediate refund transactions for all deposits that have not been confirmed in the blocks preceding the final offchain block.
Post-dated refund transactions can handle any deposits that still remain after these immediate refunds, if the depositors bothered to obtain them before creating their deposits.

Thus, the finalization and immediate/post-dated refund transactions remove any trace of the Hydrozoa head from Cardano's ledger.
The peers can manually spend (via multi-signed transactions) any other utxos that still remain at the head's native script address.

\end{document}
