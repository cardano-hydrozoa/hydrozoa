\documentclass[../hydrozoa.tex]{subfiles}
\graphicspath{{\subfix{../images/}}}
\begin{document}

\chapter{L1 protocol}%
\label{h:l1-protocol}

The Hydrozoa L1 protocol has two distinct operating regimes.
Every head is initialized by its peers in the \textbf{multisig} regime, in which all funds and state are stored at a native script address controlled by the unanimous multi-signature of its peers.
Deposits into the head are created by sending utxos to this address, which are either absorbed into the head's treasury or refunded to depositors by multi-signed transactions.
Similarly, withdrawals are settled on L1 by multi-signed transactions that send utxos out of the head's treasury.
Thus, the multisig regime's L1 script defers entirely to the L2 consensus protocol between the peers, which it perceives as a black box mechanism that produces multi-signed transactions to modify the L1 state.

Ideally, a head spends its entire life cycle in the multisig regime, right up to the final transaction that withdraws all remaining funds out of the head and de-initializes it.
However, the multisig regime may stall if the peers stop multi-signing transactions before reaching finalization.

In this case, the head moves to the \textbf{rule-based} regime.
In this regime, unabsorbed deposits are refunded and control over the head's treasury transfers to a suite of Plutus scripts. 
These scripts arbitrate the peers' dispute about the latest L2 block confirmed by the L2 consensus protocol before it stalled.
Once the dispute is resolved, the Plutus scripts manage withdrawals from the head's treasury based on that resolved L2 block.

\end{document}
