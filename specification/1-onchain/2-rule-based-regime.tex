\documentclass[../hydrozoa.tex]{subfiles}
\graphicspath{{\subfix{../images/}}}
\begin{document}

\section{Rule-based regime}%
\label{h:rule-based-regime}%
\newcommand{\disputeTokenNum}[0]{3477}
If the peers stop responding to each other in the offchain consensus protocol, the rule-based regime ensures that their funds can still be withdrawn from the Hydrozoa treasury according to the latest confirmed snapshot.
It is less cost-efficient and more complex than the multisig regime, but it serves as a fallback when offchain consensus breaks down.

In the rule-based regime, the Hydrozoa head's treasury is managed by a suite of Plutus scripts:
\begin{itemize}
  \item The dispute resolution scripts (\cref{h:rule-based-dispute-resolution}) arbitrate the peers' dispute about the latest offchain snapshot confirmed by the offchain consensus protocol before it stalled.
  \item The treasury spending validator (\cref{h:rule-based-treasury}) awaits resolution and then manages withdrawals from the head's treasury based on the resolved snapshot.
\end{itemize}

The rule-based regime does not manage any unabsorbed deposits left over from the multisig regime.
Depositors can retrieve them with post-dated refund transactions (\cref{h:multisig-refund}) from the multisig regime.

\subsection{Utxo state}%
\label{h:rule-based-utxo-state}

\subsubsection{Treasury state}%
\label{h:rule-based-treasury-state}
In the rule-based regime, the head's treasury utxo is at the head's Plutus-based treasury address (\cref{h:rule-based-treasury}).
It holds the head's beacon token and all treasury funds from the multisig regime (\cref{h:multisig-utxo-state}).

The treasury utxo must have the following datum type, which defines its unresolved and resolved states in the rule-based regime:
\begin{equation*}
\begin{split}
  \T{RuleBasedTreasuryDatum} \coloneq&\;
    \T{Resolved} \; \T{ResolvedDatum} \\\mid&\;
    \T{Unresolved} \; \T{UnresolvedDatum} \\
  \T{ResolvedDatum} \coloneq&\;\left\{
    \begin{array}{lll}
      \T{activeUtxos}  &::& \mathcal{RH}_{32} \; \T{UtxoSet} \\
      \T{version} &::& (\T{Int}, \T{Int}) \\
      \T{params} &::& \mathcal{H}_{32} \; \T{HeadParams}
    \end{array}\right\} \\
  \T{UnresolvedDatum} \coloneq&\;\left\{
    \begin{array}{lll}
      \T{disputeId} &::& \T{AssetName} \\
      \T{nPeers} &::& \T{Int} \\
      \T{params} &::& \mathcal{H}_{32} \; \T{HeadParams}
    \end{array}\right\}
\end{split}
\end{equation*}

The resolved treasury's datum is the same as the multisig treasury's datum, except that the single-integer \code{majorVersion} field is replaced by the full \code{version} field, which includes both the major and minor version numbers of the resolved snapshot.

The unresolved treasury's datum fields are interpreted as follows:
\begin{description}
  \item[dispute ID.] A unique identifier used as the asset name of the dispute tokens (\cref{h:rule-based-dispute-resolution-state}).
  \item[n peers.] The number of peers in the head, each of which gets a single vote in the dispute.
  \item[params.] A 32-byte Blake2b-256 hash (\inlineColored{$\mathcal{H}_{32}$}) of the head's offchain consensus parameters (\cref{h:offchain-consensus-protocol}).
\end{description}

\subsubsection{Dispute resolution state}%
\label{h:rule-based-dispute-resolution-state}
Besides the treasury utxo, the head's state in the rule-based regime also includes the vote utxos for dispute resolution.
Each of these utxos is at the head's Plutus-based dispute resolution address and holds one or more of the dispute's fungible tokens:
\begin{itemize}
  \item The policy ID corresponds to the head's native script (in minting policy form) (\cref{h:multisig-regime}).
  \item The first four bytes of the asset name are equal to the CIP-67
    \citep{AlessandroKonradThomasVellekoopCIP67AssetName2022}
    prefix for class label \inlineColored{\disputeTokenNum}, which corresponds to Hydrozoa head dispute tokens.%
    \footnote{We chose \inlineColored{\disputeTokenNum} because it spells ``DISP'' (short for dispute) on a phone dial pad.}
  \item The last 28 bytes of the asset name are equal to the Blake2b-224 hash (\inlineColored{$\mathcal{H}_{28}$}) of the multisig treasury utxo spent in the head's transition to the rule-based regime (\cref{h:rule-based-transition}).
\end{itemize}

Each vote utxo has the following datum type:
\begin{equation*}
\begin{split}
  \T{VoteDatum} &\coloneq \left\{
    \begin{array}{lll}
      \T{key}  &::& \T{Word64} \\
      \T{link} &::& \T{Word64} \\
      \T{deadline} &::& \T{PosixTime} \\
      \T{peer} &::& \T{Maybe} \; \T{PubKeyHash} \\
      \T{vote} &::& \T{Maybe} \; \T{Vote}
    \end{array}\right\}\\
  \T{Vote} &\coloneq \left\{
    \begin{array}{lll}
      \T{activeUtxos} &::& \mathcal{RH}_{32} \; \T{UtxoSet} \\
      \T{version} &::& (\T{Int}, \T{Int})
    \end{array}\right\}
\end{split}
\end{equation*}

These fields are interpreted as follows:
\begin{description}
  \item[key.] An unsigned integer, represented by a 64-bit word, that uniquely identifies the vote utxo.
  \item[link.] An unsigned integer, represented by a 64-bit word, that uniquely references another vote utxo by its \code{key}.
  \item[deadline.] A POSIX time before which votes can be cast in the dispute.
  \item[peer.] The public-key hash of the peer (if any) who can cast the vote in this utxo. It is only empty for the default vote utxo (with \code{key} 0), which contains a vote that is automatically cast at the transition to the rule-based regime (\cref{h:rule-based-transition}).
  \item[vote.] The vote (if any) cast in this vote utxo.
  \item[active utxos.] A 32-byte Merkle root hash (\inlineColored{$\mathcal{RH}_{32}$}) of the offchain ledger's active utxo set, as of the snapshot voted by the peer.
  \item[version.] The full version of the snapshot voted by the peer.
\end{description}

Collectively, the dispute resolution utxos constitute a shrinking linked list (\cref{h:shrinking-linked-list}).
The voting process and list transitions are managed by the Plutus-based dispute resolution scripts (\cref{h:rule-based-dispute-resolution}).

\subsection{Transition to rule-based regime}%
\label{h:rule-based-transition}

The offchain consensus protocol pairs every multisig treasury utxo that it creates with a post-dated transaction that transitions the treasury to the rule-based regime (\cref{h:offchain-consensus-protocol}).

While offchain consensus holds, the peers store these post-dated transactions without submitting.
However, if the multisig utxo is not spent by the time its corresponding post-dated transaction becomes valid, then offchain consensus is assumed to have stalled.
In that case, every peer should trigger the transition to the rule-based regime by submitting the transaction to Cardano.

The transaction body is as follows:
\begin{enumerate}
  \item Spend the multisig treasury utxo (\code{multisigTreasury}).
  \item Let \code{headMp} be the head's native script minting policy.
  \item Let \code{peersList} be a list of the peers' public-key hashes.
  \item Let \code{deadline} be the voting deadline for this dispute, as defined by the offchain consensus for this transition to the rule-based regime.
  \item Let \code{disputeId} be the CIP-67 prefix for 3477, concatenated with the Blake2b-224 hash of the \code{multisigTreasury} output reference.
  \item Let \code{nPeers} be the number of peers in the head.
  \item Let \code{params} be the \code{params} field value of \code{multisigTreasury}.
  \item Mint \code{(nPeers + 1)} dispute tokens of \code{(headMp, disputeId)}.
  \item Create the rule-based treasury utxo (\code{ruleBasedTreasury}), containing the funds and beacon token from the \code{multisigTreasury} and the following datum:
    \begin{equation*}
    \begin{split}
      \T{initRuleBasedDatum} &\coloneq \T{Unresolved} \;\left\{
        \begin{array}{lll}
          \T{disputeId} &=& \T{disputeId} \\
          \T{nPeers} &=& \T{nPeers} \\
          \T{params} &=& \T{params}
        \end{array}\right\}
    \end{split}
    \end{equation*}
  \item Create the default vote utxo, containing one dispute token and the following datum:
    \begin{equation*}
    \begin{split}
      \T{defaultVoteDatum} &\coloneq \left\{
        \begin{array}{lll}
          \T{key}  &=& 0 \\
          \T{link} &=& 0 < \T{nPeers} \;?\; \T{Just} \; 1 : \T{Nothing} \\
          \T{deadline} &=& \T{deadline} \\
          \T{peer} &=& \T{Nothing} \\
          \T{vote} &=& \T{Just} \; \T{vote}
        \end{array}\right\}\\
      \T{vote} &\coloneq \left\{
        \begin{array}{lll}
          \T{activeUtxos} &=& \T{multisigTreasury.activeUtxos} \\
          \T{version} &=& (\T{multisigTreasury.majorVersion}, 0)
        \end{array}\right\}
    \end{split}
    \end{equation*}
  \item For each public-key hash \inlineColored{$k$} in \code{peersList}, indexed by \inlineColored{$i$} (starting with 1), create a vote utxo with one dispute token and the following datum:
    \begin{equation*}
    \begin{split}
      \T{initVoteDatum} \; i \; k \coloneq \left\{
        \begin{array}{lll}
          \T{key}  &=& i \\
          \T{link} &=& i < \T{nPeers} \;?\; \T{Just} \; (i + 1) : \T{Nothing} \\
          \T{deadline} &=& \T{deadline} \\
          \T{peer} &=& k \\
          \T{vote} &=& \T{Nothing}
        \end{array}\right\}\\
    \end{split}
    \end{equation*}
\end{enumerate}

\subsection{Dispute resolution}%
\label{h:rule-based-dispute-resolution}

% TODO Dispute resolution

% Spending validator handles voting and forwards tallying to staking validator.
% Staking validator handles tallying consecutive vote utxos:
% - before voting deadline, both vote utxos being tallied must contain votes.
% - afterward, any consecutive vote utxos can be tallied.

\subsubsection{Voting}%
\label{h:voting}

% TODO - Vote
% Vote before the voting deadline.

% Spending validator

\subsubsection{Tallying}%
\label{h:tallying}

% TODO - Tally
% Remove a node if both it and its parent contain a vote, or if the transaction validity starts at or after the voting deadline.

% Spending validator -> staking validator
% staking validator looks for exactly 2 inputs with dispute tokens.
% dispute ID of these tokens must match redeemer.

% TODO - Resolve
% Remove the singleton node if the treasury utxo is also spent.

% TODO - Cleanup
% Deinitialize the shrinking list, which requires all list tokens to be burned.

% TODO - Multisig override
% Do whatever you want if the transaction is multi-signed by all peers.

\subsection{Treasury}%
\label{h:rule-based-treasury}

% TODO Treasury

% TODO - Resolve
% Compare the full version number between the treasury datum and the singleton node from dispute resolution:
%   - if the singleton node's version is greater, update the treasury datum to its version number and active_utxo field.
%   - otherwise, keep the treasury datum unchanged.

% TODO - Withdraw
% - Spend the treasury utxo.
% - Output an updated treasury utxo.
% - Output a withdrawal.
% - Prove that the new treasury's active_utxo results from
%   removing the withdrawn utxo from the old treasury's active_utxo.

% TODO - Deinit
% - Spend the treasury utxo.
% - Prove that the treasury's active_utxo is empty.

% TODO - Multisig override
% Do whatever you want if the transaction is multi-signed by all peers.

\end{document}
