\documentclass[../hydrozoa.tex]{subfiles}
\graphicspath{{\subfix{../images/}}}
\begin{document}

\chapter{Offchain state}%
\label{h:offchain-state}

Each peer in a Hydrozoa head must locally keep some offchain state to ensure that the events witnessed by the peers in the offchain consensus protocol can be accurately settled onchain.
This state is also needed to participate in the offchain consensus protocol.
It consists of:
\begin{itemize}
  \item Parameters for the offchain consensus protocol (discussed in \cref{h:consensus-setup}).
  \item The offchain ledger and the events that transition its state (\cref{h:offchain-ledger}).
  \item Snapshots of the offchain ledger, which the peers periodically generate and multi-sign so that they can be settled onchain (\cref{h:offchain-snapshots}).
\end{itemize}

In principle, a peer could choose to permanently store all offchain parameters, ledger events, ledger states, and snapshots for a Hydrozoa head.
However, in practice, the offchain consensus protocol (\cref{h:offchain-consensus-protocol}) dictates the lifecycle of this data -- when it must be kept by peers and when it is safe to discard.

\end{document}
