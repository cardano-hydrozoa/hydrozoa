\documentclass[../hydrozoa.tex]{subfiles}
\graphicspath{{\subfix{../images/}}}
\begin{document}

\section{Ledger}%
\label{h:l2-ledger}

The L2 ledger state has this type:
\begin{equation*}
  \T{LedgerState} \coloneq \left\{
  \begin{array}{lll}
    \T{activeUtxos} &::& \T{UtxoSet} \\
    \T{withdrawnUtxos} &::& \T{UtxoSet}
  \end{array}\right\}
\end{equation*}
These fields are interpreted as follows:
\begin{description}
  \item[active utxos.] The set of utxos that can be spent in transactions or withdrawn.
    These utxos are created by L2 transactions and settlements.
  \item[withdrawnUtxos.] The set of utxos that have been withdrawn from \code{activeUtxos} and can be settled onchain in the multisig regime.
    L2 settlement events empty this set (\cref{h:ledger-settlement}).
    
\end{description}

Conceptually, a utxo set is a set of transaction outputs tagged with unique references to their origins.
In practice, we equivalently represent a utxo set as a map from output reference to output:
\begin{equation*}
\begin{split}
  \T{UtxoSet} &\coloneq \T{Map(OutputRef, Output)} \\
    &\coloneq \left\{
      (k_i :: \T{OutputRef}, v_i :: \T{Output})
      \;\middle|\;
      \forall i \neq j.\; k_i \neq k_j
    \right\}
\end{split}
\end{equation*}

An output reference is a tuple that uniquely identifies an output by a hash of the ledger event that created it (either a transaction or a deposit) and its index among the outputs of that event:
\begin{equation*}
    \T{OutputRef} \coloneq \left\{
    \begin{array}{lll}
      \T{id} &::& \T{TxId} \\
        \T{index} &::& \T{Int}
    \end{array} \right\}
\end{equation*}

An output is a tuple describing a bundle of tokens, data, and a script that have been placed by a transaction at an address in the ledger:
\begin{equation*}
    \T{Output} \coloneq \left\{
    \begin{array}{lll}
      \T{addr} &::& \T{Address} \\
        \T{value} &::& \T{Value} \\
        \T{datum} &::& \T{Option(Data)} \\
        \T{script} &::& \T{Option(Script)}
    \end{array} \right\}
\end{equation*}
Note that this type definition of an \code{Output} does not allow datum hashes in the \code{datum} field, as they are not allowed in Hydrozoa L2 transactions (\cref{h:ledger-transaction}).

\subsection{Transaction}%
\label{h:ledger-transaction}

In the L2 ledger, every transaction spends one or more utxos in \code{activeUtxos}, replacing them with zero or more new utxos.
The rules for Hydrozoa's L2 transactions are the same as Cardano's ledger rules for onchain transactions,%
\footnote{At the time of writing, Cardano mainnet uses the Plomin hardfork's Conway-era ledger rules \citep{IntersectMBOCardanoLedgerV117402025}.}
but with the following additions:
\begin{description}
    \item[No staking or governance actions.]
      Hydrozoa's consensus protocol is not based on Ouroboros proof-of-stake, and its governance protocol is not based on Cardano's hard-fork-combinator update mechanism.
      Furthermore, Hydrozoa's onchain scripts cannot authorize arbitrary staking or governance actions on behalf of users.
      For this reason, staking and governance actions cannot be used in L2 transactions.%
      \footnote{We are aware that many dApps on Cardano rely on withdraw-zero staking validators (for performance gains), which are not available without staking actions.
        However, instead of conjuring a complex workaround, Hydrozoa will wait for the implementation of CIP-112 \citep{DiSarroCIP112ObserveScript2024} in an upcoming hardfork.
        The observer scripts of CIP-112 allow dApp developers to implement the same functionality without withdraw-zero staking validators in a more principled and performant way.}
        % TODO add minting subsection to future work section
    \item[No pre-Conway features.] Hydrozoa does not need to maintain backwards compatibility with pre-Conway eras.
      L2 utxos cannot use bootstrap addresses or Shelley addresses with Plutus versions older than Plutus V3.

      Public key hash credentials, native scripts, and Plutus scripts at or above V3 are allowed.
    \item[No datum hashes in outputs.] Hydrozoa requires all utxo datums to be inline, which avoids the need to index the datum-hash to datum map from transactions' datum witnesses.
    \item[No minting or burning of tokens.]
      Hydrozoa's onchain scripts are not authorized to mint or burn arbitrary user tokens on chain, which makes it impossible to reflect such events when L2 state is settled onchain.
      Thus, L2 transactions cannot mint or burn tokens.%
      \footnote{We are also aware that many dApps on Cardano rely on token minting and burning to authenticate their internal state.
        However, supporting this feature opens up a range of difficult issues that have not yet yielded satisfactory solutions in discussions \citep{IOGExcludePhantomTokens2025}.
        Fortunately, the Midgard team is exploring a new approach to this problem, which we may adopt in future extensions to Hydrozoa.}
    \item[Different network ID.] Hydrozoa L2 transactions and utxo addresses use a different network ID to distinguish them from their Cardano mainnet counterparts.
\end{description}
Conveniently, we can avoid forking Cardano's ledger rules library to implement these additional rules in Hydrozoa.
Instead, we express them as requirements for certain fields in transactions to always be omitted or to be restricted to a narrower selection of values.
The hydra node software rejects (with an error message) L2 transactions that violate these requirements, while peers simply ignore \code{ReqTx} messages (\cref{h:consensus-transaction}) that broadcast such transactions.

In the following type listing:
\begin{itemize}
  \item Some fields are already optional in Cardano transactions, with suitable defaults provided when omitted. We indicate these by prefixing their field types with a question mark (\code{?}).
  \item We indicate fields that must be omitted in Hydrozoa L2 transactions by prefixing their field names with an empty-set symbol (\code{$\varnothing$}).
\end{itemize}
\begingroup
\allowdisplaybreaks
\begin{align*}
    \T{Tx} \coloneq\;& \left\{
    \begin{array}{lll}
      \T{body} &::& \T{TxBody} \\
        \T{wits} &::& \T{TxWits} \\
        \T{is\_valid} &::& \T{Bool} \\
        \T{auxiliary\_data} &::& \quad?\;\T{TxMetadata}
    \end{array} \right\} \\
    \T{TxBody} \coloneq\;& \left\{
    \begin{array}{lll}
      \T{spend\_inputs} &::& \T{Set(OutputRef)} \\
        \T{collateral\_inputs} &::& \quad?\;\T{Set(OutputRef)} \\
        \T{reference\_inputs} &::& \quad?\;\T{Set(OutputRef)} \\
        \T{outputs} &::& \T{[Output]} \\
        \T{collateral\_return} &::& \quad?\;\T{Output} \\
        \T{total\_collateral} &::& \quad?\;\T{Coin} \\
        \varnothing\;\T{certificates} &::& \quad?\;\T{[ Set(Certificate) ]} \\
        \varnothing\;\T{withdrawals} &::& \quad?\;\T{Map(RewardAccount, Coin)} \\
        \T{fee} &::& \T{Coin} \\
        \T{validity\_interval} &::& \quad?\;\T{ValidityInterval} \\
        \T{required\_signer\_hashes} &::& \quad?\;\T{[VKeyCredential]} \\
        \varnothing\;\T{mint} &::& \quad?\;\T{Value} \\
        \T{script\_integrity\_hash} &::& \quad?\;\T{Hash} \\
        \T{auxiliary\_data\_hash} &::& \quad?\;\T{Hash} \\
        \T{network\_id} &::& \quad?\;\T{Network} \\
        \varnothing\;\T{voting\_procedures} &::& \quad?\;\T{VotingProcedures} \\
        \varnothing\;\T{proposal\_procedures} &::& \quad?\;\T{Set(ProposalProcedure)} \\
        \varnothing\;\T{current\_treasury\_value} &::& \quad?\;\T{Coin} \\
        \varnothing\;\T{treasury\_donation} &::& \quad?\;\T{Coin}
    \end{array} \right\}\\
    \T{TxWits} \coloneq\;& \left\{
    \begin{array}{lll}
      \T{addr\_tx\_wits} &::& \quad?\;\T{Set(VKey, Signature, VKeyHash)} \\
        \varnothing\;\T{boot\_addr\_tx\_wits} &::& \quad?\;\T{Set(BootstrapWitness)} \\
        \T{script\_tx\_wits} &::& \quad?\;\T{Map(ScriptHash, Script)} \\
        \varnothing\;\T{data\_tx\_wits} &::& \quad?\;\T{TxDats} \\
        \T{redeemer\_tx\_wits} &::& \quad?\;\T{Redeemers}
    \end{array} \right\}\\
    \T{Script} \coloneq\;& \T{TimelockScript}(\T{Timelock}) \\
                          \mid\;& \T{PlutusScript}(\T{PlutusVersion},\T{PlutusBinary}) \\
    \T{PlutusVersion} \coloneq\;& \T{PlutusV3}
\end{align*}
\endgroup

\subsection{Withdrawal}%
\label{h:ledger-withdrawal}

A withdrawal is an L2 ledger event that spends one or more utxos in \code{activeUtxos} and creates identical utxos (with the same output references) in \code{withdrawnUtxos}.
The L2 ledger rules for withdrawals are similar to transactions, but with the following modifications:
\begin{description}
  \item[No outputs.] Withdrawals cannot create utxos in \code{activeUtxos}.
  \item[No fees.] Withdrawals do not pay transaction fees.
  \item[No collateral.] Withdrawals do not provide collateral for phase-2 validation.
  \item[No metadata.] Withdrawals cannot contain transaction metadata.
\end{description}

Similar to L2 transactions, we can express the modified rules for withdrawals (relative to Cardano ledger rules) as requirements for certain fields to be omitted or restricted:%
\footnote{Here, ``omitting'' the \inlineColored{outputs} and \inlineColored{fee} fields means setting them to an empty list and zero, respectively.}
\begingroup
\allowdisplaybreaks
\begin{align*}
    \T{Tx} \coloneq\;& \left\{
    \begin{array}{lll}
      \T{body} &::& \T{TxBody} \\
        \T{wits} &::& \T{TxWits} \\
        \T{is\_valid} &::& \T{Bool} \\
        \varnothing\;\T{auxiliary\_data} &::& \quad?\;\T{TxMetadata}
    \end{array} \right\} \\
    \T{TxBody} \coloneq\;& \left\{
    \begin{array}{lll}
      \T{spend\_inputs} &::& \T{Set(OutputRef)} \\
        \varnothing\;\T{collateral\_inputs} &::& \quad?\;\T{Set(OutputRef)} \\
        \T{reference\_inputs} &::& \quad?\;\T{Set(OutputRef)} \\
        \varnothing\;\T{outputs} &::& \T{[Output]} \\
        \varnothing\;\T{collateral\_return} &::& \quad?\;\T{Output} \\
        \varnothing\;\T{total\_collateral} &::& \quad?\;\T{Coin} \\
        \varnothing\;\T{certificates} &::& \quad?\;\T{[ Set(Certificate) ]} \\
        \varnothing\;\T{withdrawals} &::& \quad?\;\T{Map(RewardAccount, Coin)} \\
        \varnothing\;\T{fee} &::& \T{Coin} \\
        \T{validity\_interval} &::& \quad?\;\T{ValidityInterval} \\
        \T{required\_signer\_hashes} &::& \quad?\;\T{[VKeyCredential]} \\
        \varnothing\;\T{mint} &::& \quad?\;\T{Value} \\
        \T{script\_integrity\_hash} &::& \quad?\;\T{Hash} \\
        \T{auxiliary\_data\_hash} &::& \quad?\;\T{Hash} \\
        \T{network\_id} &::& \quad?\;\T{Network} \\
        \varnothing\;\T{voting\_procedures} &::& \quad?\;\T{VotingProcedures} \\
        \varnothing\;\T{proposal\_procedures} &::& \quad?\;\T{Set(ProposalProcedure)} \\
        \varnothing\;\T{current\_treasury\_value} &::& \quad?\;\T{Coin} \\
        \varnothing\;\T{treasury\_donation} &::& \quad?\;\T{Coin}
    \end{array} \right\}\\
    \T{TxWits} \coloneq\;& \left\{
    \begin{array}{lll}
      \T{addr\_tx\_wits} &::& \quad?\;\T{Set(VKey, Signature, VKeyHash)} \\
        \varnothing\;\T{boot\_addr\_tx\_wits} &::& \quad?\;\T{Set(BootstrapWitness)} \\
        \T{script\_tx\_wits} &::& \quad?\;\T{Map(ScriptHash, Script)} \\
        \varnothing\;\T{data\_tx\_wits} &::& \quad?\;\T{TxDats} \\
        \T{redeemer\_tx\_wits} &::& \quad?\;\T{Redeemers}
    \end{array} \right\}\\
    \T{Script} \coloneq\;& \T{TimelockScript}(\T{Timelock}) \\
                          \mid\;& \T{PlutusScript}(\T{PlutusVersion},\T{PlutusBinary}) \\
    \T{PlutusVersion} \coloneq\;& \T{PlutusV3}
\end{align*}
\endgroup

Unfortunately, Cardano's ledger rules do not allow the \code{fee} to be zero, and they do not allow the \code{collateral\_inputs}, \code{collateral\_return}, and \code{total\_collateral} fields to be omitted in transactions containing Plutus scripts.
Thus, when validating L2 withdrawals, we must \emph{deactivate} the following Cardano ledger rules:
\begin{itemize}
  \item Minimum fee amount
  \item Minimum collateral amount
  \item Non-empty collateral inputs (if Plutus scripts used)
  \item Collateral balancing:
    \begin{equation*}
      \sum \T{lovelace} (\T{collateral\_inputs}) =
      \T{lovelace} (\T{collateral\_return}) + \T{total\_collateral}
    \end{equation*}
  \item Transaction balancing:
    \begin{equation*}
      \sum \T{value} (\T{inputs}) + \T{mint} =
      \sum \T{value} (\T{outputs}) + \T{fee}
    \end{equation*}
\end{itemize}

Theoretically, we could effectively deactivate the minimum fee, minimum collateral, and collateral balancing rules by setting certain protocol parameters to zero.
However, since the non-empty collateral inputs and transaction balancing rules are not affected by protocol parameters, forking Cardano's ledger rules is unavoidable for L2 withdrawals.

\subsection{Settlement}%
\label{h:ledger-settlement}

L2 settlement is a ledger event that is automatically inserted into the ledger's event history immediately following the events confirmed by a major L2 block.
The effect of this event is to empty the \code{withdrawnUtxo} set and add new utxos to the \code{activeUtxo} set.

These new utxos correspond to the deposits absorbed by the settlement transaction (\cref{h:multisig-settle}) of the major L2 block, with the \code{Output} fields of each new utxo set as follows:
\begin{description}
  \item[addr.] Set equal to the \code{address} field in the corresponding onchain deposit's datum.
  \item[value.] Set equal to the value of the corresponding onchain deposit.
  \item[datum.] Set equal to the \code{datum} field in the corresponding onchain deposit's datum.
  \item[script.] Empty.
\end{description}

The \code{OutputReference} fields of each of these new utxos is as follows:
\begin{description}
  \item[id.] Set equal to the Blake2b-256 hash (\inlineColored{$\mathcal{H}_{32}$}) of the transaction ID of the major L2 block's settlement transaction.
  \item[index.] Set equal to the onchain deposit's index among the absorbed deposits of the major L2 block, in ascending order of onchain \code{OutputReference}, starting from zero.
\end{description}

\end{document}
