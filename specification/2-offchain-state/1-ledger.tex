\documentclass[../hydrozoa.tex]{subfiles}
\graphicspath{{\subfix{../images/}}}
\begin{document}

\section{Ledger}%
\label{h:offchain-ledger}

The offchain ledger state has this type:
\begin{equation*}
  \T{LedgerState} \coloneq \left\{
  \begin{array}{lll}
    \T{activeUtxos} &::& \T{UtxoSet} \\
    \T{withdrawnUtxos} &::& \T{UtxoSet}
  \end{array}\right\}
\end{equation*}
These fields are interpreted as follows:
\begin{description}
  \item[active utxos.] The set of utxos that can be spent in transactions or withdrawn.
    These utxos are created by offchain transactions and settlements.
  \item[withdrawnUtxos.] The set of utxos that have been withdrawn from \code{activeUtxos} and can be settled onchain in the multisig regime.
    Offchain settlement events empty this set (\cref{h:ledger-settlement}).
    
\end{description}

Conceptually, a utxo set is a set of transaction outputs tagged with unique references to their origins.
In practice, we equivalently represent a utxo set as a map from output reference to output:
\begin{equation*}
\begin{split}
  \T{UtxoSet} &\coloneq \T{Map(OutputRef, Output)} \\
    &\coloneq \left\{
      (k_i :: \T{OutputRef}, v_i :: \T{Output})
      \;\middle|\;
      \forall i \neq j.\; k_i \neq k_j
    \right\}
\end{split}
\end{equation*}

An output reference is a tuple that uniquely identifies an output by a hash of the ledger event that created it (either a transaction or a deposit) and its index among the outputs of that event (always zero for deposits):
\begin{equation*}
    \T{OutputRef} \coloneq \left\{
    \begin{array}{lll}
      \T{id} &::& \T{TxId} \\
        \T{index} &::& \T{Int}
    \end{array} \right\}
\end{equation*}

An output is a tuple describing a bundle of tokens, data, and a script that have been placed by a transaction at an address in the ledger:
\begin{equation*}
    \T{Output} \coloneq \left\{
    \begin{array}{lll}
      \T{addr} &::& \T{Address} \\
        \T{value} &::& \T{Value} \\
        \T{datum} &::& \T{Option(Data)} \\
        \T{script} &::& \T{Option(Script)}
    \end{array} \right\}
\end{equation*}
Note that this type definition of an \code{Output} does not allow datum hashes in the \code{datum} field, as they are not allowed in Hydrozoa offchain transactions (\cref{h:ledger-transaction}).

\subsection{Transaction}%
\label{h:ledger-transaction}

In the offchain ledger, every transaction spends one or more utxos in \code{activeUtxos}, replacing them with zero or more new utxos.
The rules for Hydrozoa's offchain transactions are the same as Cardano's ledger rules for onchain transactions,%
\footnote{At the time of writing, Cardano mainnet uses the Plomin hardfork's Conway-era ledger rules \citep{IntersectMBOCardanoLedgerV117402025}.}
but with the following additions:
\begin{description}
    \item[No staking or governance actions.]
      Hydrozoa's consensus protocol is not based on Ouroboros proof-of-stake, and its governance protocol is not based on Cardano's hard-fork-combinator update mechanism.
      Furthermore, Hydrozoa's onchain scripts cannot authorize arbitrary staking or governance actions on behalf of users.
      For this reason, staking and governance actions cannot be used in offchain transactions.%
      \footnote{We are aware that many dApps on Cardano rely on withdraw-zero staking validators (for performance gains), which are not available without staking actions.
        However, instead of conjuring a complex workaround, Hydrozoa will wait for the implementation of CIP-112 \citep{DiSarroCIP112ObserveScript2024} in an upcoming hardfork.
        The observer scripts of CIP-112 allow dApp developers to implement the same functionality without withdraw-zero staking validators in a more principled and performant way.}
        % TODO add minting subsection to future work section
    \item[No pre-Conway features.] Hydrozoa does not need to maintain backwards compatibility with pre-Conway eras.
      Offchain utxos cannot use bootstrap addresses or Shelley addresses with Plutus versions older than Plutus V3.

      Public key hash credentials, native scripts, and Plutus scripts at or above V3 are allowed.
    \item[No datum hashes in outputs.] Hydrozoa requires all utxo datums to be inline, which avoids the need to index the datum-hash to datum map from transactions' datum witnesses.
    \item[No minting or burning of tokens.]
      Hydrozoa's onchain scripts are not authorized to mint or burn arbitrary user tokens on chain, which makes it impossible to reflect such events when offchain state is settled onchain.
      Thus, offchain transactions cannot mint or burn tokens.%
      \footnote{We are also aware that many dApps on Cardano rely on token minting and burning to authenticate their internal state.
        However, supporting this feature opens up a range of difficult issues that have not yet yielded satisfactory solutions in discussions \citep{IOGExcludePhantomTokens2025}.
        Fortunately, the Midgard team is exploring a new approach to this problem, which we may adopt in future extensions to Hydrozoa.}
    \item[Different network ID.] Hydrozoa offchain transactions and utxo addresses use a different network ID to distinguish them from their Cardano mainnet counterparts.
\end{description}
Conveniently, we can avoid forking Cardano's ledger rules library to implement these additional rules in Hydrozoa.
Instead, we express them as requirements for certain fields in transactions to always be omitted or to be restricted to a narrower selection of values.
The hydra node software rejects (with an error message) offchain transactions that violate these requirements, while peers simply ignore \code{ReqTx} messages (\cref{h:consensus-transaction}) that broadcast such transactions.

In the following type listing:
\begin{itemize}
  \item Some fields are already optional in Cardano transactions, with suitable defaults provided when omitted. We indicate these by prefixing their field types with a question mark (\code{?}).
  \item We indicate fields that must be omitted in Hydrozoa offchain transactions by prefixing their field names with an empty-set symbol (\code{$\varnothing$}).
\end{itemize}
\begingroup
\allowdisplaybreaks
\begin{align*}
    \T{Tx} \coloneq\;& \left\{
    \begin{array}{lll}
      \T{body} &::& \T{TxBody} \\
        \T{wits} &::& \T{TxWits} \\
        \T{is\_valid} &::& \T{Bool} \\
        \T{auxiliary\_data} &::& \quad?\;\T{TxMetadata}
    \end{array} \right\} \\
    \T{TxBody} \coloneq\;& \left\{
    \begin{array}{lll}
      \T{spend\_inputs} &::& \T{Set(OutputRef)} \\
        \T{collateral\_inputs} &::& \quad?\;\T{Set(OutputRef)} \\
        \T{reference\_inputs} &::& \quad?\;\T{Set(OutputRef)} \\
        \T{outputs} &::& \T{[Output]} \\
        \T{collateral\_return} &::& \quad?\;\T{Output} \\
        \T{total\_collateral} &::& \quad?\;\T{Coin} \\
        \varnothing\;\T{certificates} &::& \quad?\;\T{[ Set(Certificate) ]} \\
        \varnothing\;\T{withdrawals} &::& \quad?\;\T{Map(RewardAccount, Coin)} \\
        \T{fee} &::& \T{Coin} \\
        \T{validity\_interval} &::& \quad?\;\T{ValidityInterval} \\
        \T{required\_signer\_hashes} &::& \quad?\;\T{[VKeyCredential]} \\
        \T{mint} &::& \quad?\;\T{Value} \\
        \T{script\_integrity\_hash} &::& \quad?\;\T{Hash} \\
        \T{auxiliary\_data\_hash} &::& \quad?\;\T{Hash} \\
        \T{network\_id} &::& \quad?\;\T{Network} \\
        \varnothing\;\T{voting\_procedures} &::& \quad?\;\T{VotingProcedures} \\
        \varnothing\;\T{proposal\_procedures} &::& \quad?\;\T{Set(ProposalProcedure)} \\
        \varnothing\;\T{current\_treasury\_value} &::& \quad?\;\T{Coin} \\
        \varnothing\;\T{treasury\_donation} &::& \quad?\;\T{Coin}
    \end{array} \right\}\\
    \T{TxWits} \coloneq\;& \left\{
    \begin{array}{lll}
      \T{addr\_tx\_wits} &::& \quad?\;\T{Set(VKey, Signature, VKeyHash)} \\
        \varnothing\;\T{boot\_addr\_tx\_wits} &::& \quad?\;\T{Set(BootstrapWitness)} \\
        \T{script\_tx\_wits} &::& \quad?\;\T{Map(ScriptHash, Script)} \\
        \varnothing\;\T{data\_tx\_wits} &::& \quad?\;\T{TxDats} \\
        \T{redeemer\_tx\_wits} &::& \quad?\;\T{Redeemers}
    \end{array} \right\}\\
    \T{Script} \coloneq\;& \T{TimelockScript}(\T{Timelock}) \\
                          \mid\;& \T{PlutusScript}(\T{PlutusVersion},\T{PlutusBinary}) \\
    \T{PlutusVersion} \coloneq\;& \T{PlutusV3}
\end{align*}
\endgroup

\subsection{Withdrawal}%
\label{h:ledger-withdrawal}

\subsection{Settlement}%
\label{h:ledger-settlement}

\end{document}
