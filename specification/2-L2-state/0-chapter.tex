\documentclass[../hydrozoa.tex]{subfiles}
\graphicspath{{\subfix{../images/}}}
\begin{document}

\chapter{L2 state}%
\label{h:l2-state}

Each peer in a Hydrozoa head must locally keep some L2 state to participate in the L2 consensus protocol and to ensure that the events witnessed by the peers on L2 can be accurately settled on L1.
It consists of:
\begin{itemize}
  \item L2 ledger states and ledger events that transition between them.
  \item L2 blocks of ledger events, periodically generated by the peers to confirm the events and settle the resulting L2 ledger states on L1.
\end{itemize}

In principle, a peer could choose to permanently store all L2 ledger states, events, and blocks for a Hydrozoa head.
However, in practice, the L2 consensus protocol (\cref{h:l2-consensus-protocol}) dictates the lifecycle of these data objects -- when they must be kept by peers and when they are safe to discard.

Each peer must also locally keep certain parameters and state associated with the actual operation of the L2 consensus protocol, but we describe them separately in \cref{h:l2-consensus-protocol}.

\end{document}
